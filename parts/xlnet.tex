\section{XLNET}
Google Brain ve Carnegie Mellon University tarafından geliştirilmiştir. Transformer-XL mimarisine dayanır. Autoregressive (kendi kendine geriye doğru tahmin yapma) modelleme ve MLM (masked language modelling) tekniklerini birleştirerek kullanır. Uzun metin bağımlılıklarını etkili bir şekilde modelleyebilmek için geliştirilmiştir. İçerisinde;

\begin{itemize}
	\item \textbf{Positional Encoding}: Girdi dizisinin pozisyon bilgilerini ekler.
	\item \textbf{Segment Embeddings}: Metni parçalarına ayırarak her parçaya bir segment kimliği atar.
	\item \textbf{Self-Attention Mechanism}: Her kelimenin diğer kelimeler ile ilişkisini öğrenir.
	\item \textbf{Autoregressive Factorization}: Girdi dizisini çeşitli sıralamalara göre modelleyerek daha fazla bilgi öğrenir.
	\item \textbf{Two-Stream Self-Attentions}: Her pozisyonda hem içerik hem de pozisyon bilgilerini öğrenir. 
\end{itemize}

\subsection{Çalışma Adımları}
\begin{enumerate}
	\item Girdi dizisi, kelime gömmeleri ve pozisyon bilgileri eklenir.
	\item Girdi dizisi, çeşitli sıralamalara göre ayrıştırılır ve her sıralama üzerinden model eğitilir.
	\item Content Stream, her pozisyondaki kelimenin içerik bilgilerini işler. Query Stream, her pozisyondeki kelimenin hem içerik hem de pozisyon bilgilerini işler.
	\item Girdi dizisinin farklı permütasyonları üzerinden dil modelleme yapılır, bu da modeilln daha geniş bir bilgi havuzundan öğrenmesini sağlar.
	\item Model, farklı permütasyonlar üzerinden eğitilerek dil modelleme görevinde daha güçlü hale getirilir.
\end{enumerate}

\newpage