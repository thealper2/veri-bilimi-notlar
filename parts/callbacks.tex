\section{Callbacks}

\subsection{Model Checkpoint}
Eğitim sırasında modelin belirli aralıklarla kaydedilmesini sağlar.

\subsubsection{Python Kodu}

\begin{lstlisting}[language=Python]
from keras.callbacks import ModelCheckpoint

checkpoint = ModelCheckpoint(filepath='best_model.h5',
							monitor='val_accuracy',
							verbose=1,
							save_best_only=True,
							mode='max')
\end{lstlisting}

\subsubsection{Hiperparametreler}
\begin{table}[h]
\centering
{\scriptsize\renewcommand{\arraystretch}{0.4}
{\resizebox*{\linewidth}{0.4\textwidth}{
\begin{tabular}{|p{2cm}|p{6cm}|}
\hline
Parametre & Açıklama \\ \hline
filepath & Kaydedilecek dosyanın yolu. \\ \hline
monitor & İzlenecek metriğin adı. \\ \hline
verbose & Çıktı durumu. \\ \hline
save\_best\_only & Sadece en iyi performans gösterdiği durumları kaydet. \\ \hline
mode & Monitor metriğinin neye göre değerlendirileceği. "max" veya "min". \\ \hline
period & Kaydetme aralığı. \\ \hline

\end{tabular}
}}}
\end{table}

\newpage

\subsection{BackupAndRestore}
Eğitim sırasında modelin yedeklerini alarak ve geri yükleyerek eğitim sırasında oluşabilecek herhangi bir kesintiden sonra eğitimin devam etmesini sağlar.

\subsubsection{Python Kodu}

\begin{lstlisting}[language=Python]
from keras.callbacks import BackupAndRestore

backup = BackupAndRestore(backup_dir,
						 save_freq="epoch")
\end{lstlisting}

\subsubsection{Hiperparametreler}
\begin{table}[h]
\centering
{\scriptsize\renewcommand{\arraystretch}{0.4}
{\resizebox*{\linewidth}{0.1\textwidth}{
\begin{tabular}{|p{2cm}|p{6cm}|}
\hline
Parametre & Açıklama \\ \hline
backup\_dir & Yedekleme dizini. \\ \hline
frequency & Yedekleme sıklığı. \\ \hline

\end{tabular}
}}}
\end{table}

\newpage

\subsection{TensorBoard}
Tensorflow'un görselleştirme aracıdır ve modelin performansını izlemek, hata analizi yapmak ve eğitim sürecini anlamak için kullanılır.

\subsubsection{Python Kodu}

\begin{lstlisting}[language=Python]
from keras.callbacks import Tensorboard

tensorboard_callback = TensorBoard(log_dir='./logs')
\end{lstlisting}

\subsubsection{Hiperparametreler}
\begin{table}[h]
\centering
{\scriptsize\renewcommand{\arraystretch}{0.4}
{\resizebox*{\linewidth}{0.2\textwidth}{
\begin{tabular}{|p{2cm}|p{6cm}|}
\hline
Parametre & Açıklama \\ \hline
log\_dir & Log kaydedileceği dizin. \\ \hline
histogram\_freq & Histogram güncelleme sıklığı. \\ \hline
write\_graph & Model grafiğinin tensorboard'da gösterilmesi \\ \hline
write\_images & Modelin görsel tanımının tensorboard'da gösterilmesi. \\ \hline

\end{tabular}
}}}
\end{table}

\newpage

\subsection{EarlyStopping}
Belirli bir metriği izleyerek iyileşmediği durumlarda eğitimi durdurur. Bu, overfitting durumlarında eğitimi sonlandırmak için kullanılır.

\subsubsection{Python Kodu}

\begin{lstlisting}[language=Python]
from keras.callbacks import EarlyStopping

early_stopping = EarlyStopping(monitor='val_loss', patience=5)
\end{lstlisting}

\subsubsection{Hiperparametreler}
\begin{table}[h]
\centering
{\scriptsize\renewcommand{\arraystretch}{0.4}
{\resizebox*{\linewidth}{0.2\textwidth}{
\begin{tabular}{|p{2cm}|p{6cm}|}
\hline
Parametre & Açıklama \\ \hline
monitor & Takip edilen metrik. \\ \hline
patience & İzlenecek epoch sayısı. \\ \hline
mode & Değerlendirme türü "min" veya "max". \\ \hline
baseline & Eğitimin durdurulması için belirli bir referans değeri. \\ \hline
restore\_best\_weights & En iyi modelin ağırlıklarını geri yükleme. \\ \hline

\end{tabular}
}}}
\end{table}

\newpage

\subsection{LearningRateScheduler}
Belirli bir metriği izleyerek iyileşmediği durumlarda eğitimi durdurur. Bu, overfitting durumlarında eğitimi sonlandırmak için kullanılır.

\subsubsection{Python Kodu}

\begin{lstlisting}[language=Python]
from keras.callbacks import LearningRateScheduler

def lr_schedule(epoch, lr): 
	if epoch < 10:
		return lr
	else:
		return lr * np.exp(-0.5)

lr_scheduler = LearningRateScheduler(schedule=lr_scheduler)
\end{lstlisting}

\newpage

\subsection{ReduceLROnPlateau}
Eğitim sırasında belirli bir metriği iyileşmediği durumlarda öğrenme oranını düşürmek için kullanılır.

\subsubsection{Python Kodu}

\begin{lstlisting}[language=Python]
from keras.callbacks import ReduceLROnPlateau

lr_plat = ReduceLROnPlateua(monitor='val_loss', patience=3, factor=0.2)
\end{lstlisting}

\subsubsection{Hiperparametreler}
\begin{table}[h]
\centering
{\scriptsize\renewcommand{\arraystretch}{0.4}
{\resizebox*{\linewidth}{0.2\textwidth}{
\begin{tabular}{|p{2cm}|p{6cm}|}
\hline
Parametre & Açıklama \\ \hline
monitor & Takip edilecek metrik adı. \\ \hline
factor & Öğrenme oranını azaltmak için çarpan. \\ \hline
patience & İzlenecek epoch sayısı. \\ \hline
mode & Değerlendirme türü. \\ \hline
cooldown & Öğrenme oranını azalttıktan sonra bekleyeceği epoch sayısı. \\ \hline

\end{tabular}
}}}
\end{table}

\newpage

\subsection{CSVLogger}
Eğitim sırasında modelin metriklerini bir CSV dosyasına kaydeder.

\subsubsection{Python Kodu}

\begin{lstlisting}[language=Python]
from keras.callbacks import CSVLogger

csv_log = CSVLogger(filename='file.csv', seperator=',')
\end{lstlisting}

\newpage