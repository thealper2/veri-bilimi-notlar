\section{REALNVP (Real-valued Non-Volume Preserving Transformation)}

REALNVP (Real-valued Non-Volume Preserving Transformation), Normalizing Flow ailesine ait bir generative modeldir. Bu model, karmaşık veri dağılımlarını modellemek için invertible (tersinir) dönüşümler kullanır. Normalizing Flow'da olduğu gibi, basit bir olasılık dağılımı (genellikle bir Gauss dağılımı) karmaşık bir dağılıma dönüştürülür. REALNVP, bu dönüşümleri gerçekleştirirken özellikle hacim koruma zorunluluğu olmayan dönüşümler (non-volume preserving transformations) kullanarak esneklik sağlar. Affine Coupling Layers, REALNVP'nin temel yapı taşıdır. Bu katmanlar, verileri iki gruba ayırır ve bir grup sabit kalırken diğer grup dönüşüme uğrar. Bu sayede dönüşümlerin tersinirliği korunur. Affine Coupling Layers, bir gruptaki verileri scale (s) ve translation (t) parametrelerine göre ölçekler ve kaydırır. Bu parametreler, model tarafından öğrenilir. Veriler, Masked Layers ile iki grupta maskelenir. Bir grup olduğu gibi bırakılırken, diğer grup dönüşümlerden geçer.

\subsection{Çalışma Adımları}

\[ y_1 = x_1 \]
\[ y_2 = x_2 * exp(s(x_1)) + t(x_1) \]

\begin{enumerate}
    \item Genellikle bir Gauss dağılımı gibi basit bir dağılımdan başlar. Bu dağılımdan rastgele örnekler alınır.
    \item Veriler iki gruba ayrılır. Bir grup sabit kalır (identity transform uygulanır) ve diğer grup affine dönüşümden geçer.
    \item Dönüşümler tersinir olduğu için, öğrenilen karmaşık dağılımdan yeni örnekler üretmek mümkündür. Ters dönüşümler, aynı formüllerin ters yönde uygulanmasıyla gerçekleştirilir.
    \item Dönüşüm sırasında Jacobian determinantı hesaplanarak olasılık yoğunluğu güncellenir. Affine dönüşüm sırasında Jacobian determinantı yalnızca $s(x_1)$'e bağlıdır, bu da hesaplamaları basitleştirir.
    \item Karmaşık veri uzayına dönüşüm yapıldıktan sonra, modelden yeni örnekler üretilebilir. Bu örnekleme süreci, tersinir dönüşümlerin uygulanmasıyla gerçekleştirilir.
\end{enumerate}



\newpage