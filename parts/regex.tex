\section{Düzenli İfadeler (Regex)}
Karakter dizileri içinde belirli örüntüleri aramayı sağlayan bir dildir. 1956 yılında Stephen C. Kleene tarafından formel bir model olarak sunulmuştur. Herhangi bir düzenli ifade doğrudan NFSA'ya buradan da DFSA'ya dönüştürlebilir.

En basit düzenli ifadeler karakterlerin sıralı biçimde dizilmesiyle oluşur. Düzenli ifadeler büyük-küçük harfe duyarlıdır.

\verb|^| işareti satır başlarını \verb|$| işareti satır sonlarını gösterir. \verb|^$| Satır başından hemen sonra gelen satır sonlarını bulur yani boş satırları bulur.

Örnekler
\begin{itemize}
    \item \verb|^Cansu|: Satır başlarındaki "Cansu"ları bulur.
    \item \verb|Cansu$|: Satır sonlarındaki "Cansu"ları bulur.
    \item \verb|^Cansu$|: Hem satır sonunda hem satır başında hem de sadece "Cansu" olanları bulur.
\end{itemize}

\verb|[...]| ifadesi REGEX'te karakter sınıfı olarak bilinir. \verb|[-]| eğer içerisinde tire işareti varsa bu iki karakter arasında aralık belirtir. \verb|[^...]| ifadesi ise uymayan kriterleri bulur.

Örnekler
\begin{itemize}
    \item \verb|makin[ae]|: makine ve makina geçenleri bulur.
    \item \verb|ke[dl]i|: keli, kedi geçenleri bulur.
    \item \verb|[Cc]ansu|: Cansu ve cansu geçenleri bulur.
    \item \verb|<h[1-6]>|: HTML'deki tüm h etiketlerini bulur.
    \item \verb|[^3-8]|: 3 ve 8 arasında olmayanları bulur.
\end{itemize}

\verb|.| ifadesi joker olarak düşünülebilir. Herhangi bir karakterle eşleşebilir. Yazı içerisindeki noktaları bulmak için noktanin önüne \\ işareti koyulur. \verb|?| karakteri kendinden önce gelen karakterin seçimlik olduğunu belirtir.

Örnekler
\begin{itemize}
    \item \verb|s.cak|: sicak, sıcak gibi kelimeleri bulur.
    \item \verb|193\.168\.1\.1.|: 193.168.1.1, 193.168.1.1a gibi kelimeleri bulur.
    \item \verb|kedileri?|: kediler veya kedileri.
    \item \verb|colou?r|: color veya colour.
\end{itemize}

\verb|*| karakteri kendinden önce gelen karakterlerin 0 veya daha fazla kere ardışık olarak tekrarlandığını belirtir. \verb|+| karakteri kendinden önce gelen karakterlerin 1 veya daha fazla kere ardışık olarak tekrarlandığını belirtir.

Örnekler
\begin{itemize}
    \item \verb|ab*c|: ac, abc, abbc, abbbc ...
    \item \verb|[0-9][0-9]*|: bir veya daha fazla sayıda ardışık rakam.
    \item \verb|[0-9]+|: bir veya daha fazla sayıda ardışık rakam.
\end{itemize}

\verb|\b| aranan ifadenin önünde veya arkasında sınırlayıcı (boşluk gibi) karakterleri sınır olarak kabul eder. \verb|\B| karakteri sınırlandırma olmayan durumu belirtir. | veya anlamına gelir. Birden fazla düzenli ifadeyi birleştirip tek bir düzenli ifade oluşturmak için kullanılır.

Örnekler
\begin{itemize}
    \item \verb|\beli\b|: Önünde ve arkasında boşluk olan 'eli' ifadesini bulur.
\end{itemize}

Sayaçlar, herhangi bir karakterin ne miktarda tekrarlanabileceğini belirten ifadelerdir. \verb|{n}| kendinden nceki karakter n defa ardışık olmalıdır. \verb|{n,m}| kendinden önceki karakter n ile m aralığında ardışık olmalıdır. \verb|{n,}| kendinden önceki karakter en az n kadar ardışık olmalıdır.

Kendinden önce 4 karakter ve 3 rakam gelen "Ateşoğlu" kelimesini bulalım: \verb|[a-z]{4}[0-9]{3}Ateşoğlu|

İşlem öncelikleri (yüksekten düşüğe doğru):
\begin{enumerate}
    \item Parantez \verb|()|
    \item Sayaçlar \verb|* + ? {}|
    \item Seriler veya bağlayıcılar \verb|evler^Yarın gelecek$|
    \item Veya | (pipe)
\end{enumerate}

Özel operatörler
\begin{itemize}
    \item \verb|\d|: herhangi bir rakam
    \item \verb|\D|: rakam olmayan bir karakter
    \item \verb|\w|: alfanümerik veya boşluk karakteri
    \item \verb|\W|: alfanümerik olmayan karakter
    \item \verb|\s|: boşluk
    \item \verb|\S|: boşluk olmayan karakter
    \item \verb|\.|: nokta karakteri
    \item \verb|\*|: * (asterisk) karakteri
    \item \verb|\?|: ? karakteri
    \item \verb|\n|: newline karakteri
    \item \verb|\t|: tab karakteri
\end{itemize}

\newpage