\section{Büyük Sayılar Kanunu (Law of Large Numbers)}
Büyük Sayılar Kanunu, bir rastgele değişkenin uzun vadeli ortalamasının, yeterince büyük sayıda tekrarlandığınad, beklenen değere yakınsamasını ifade eder. Veri sayısı arttıkça elde edilen sonucun güvenilirliği artmaktadır. Örneğin bir bozuk para atma deneyinde yazı ve tura olasılığı 0.5 olmak üzere eşittir. Üst üste yazı veya tura gelebilir. Bu, birinin olasılığının diğerinden fazla olduğu anlamına gelmez. Bu bağlamda çok sayıda para atımı yapıldığında eşit sayıda yazı veya tura elde edilebilir.

Zayıf Büyük Sayılar Kanunu (Weak Law of Large Numbers), örneklem ortalamasının beklenen değere yakınsadığını, yani belirli bir hata payı içinde kalacağını belirtir.

\[ \frac{1}{n} \sum_{i=1}^{n} X_i \xrightarrow{P} \mu \]

\begin{itemize}
	\item $X_i$: i'inci bağımsız ve özdeş dağılımlı rastgele değişken.
	\item $n$: Gözlem sayısı.
	\item $\mu$: Beklenen değer.
	\item $\xrightarrow{P}$: Olasılıkla yakınsama.
\end{itemize}

Güçlü Büyük Sayılar Kanunu (Strong Law of Large Numbers), örneklem ortalamasının beklenen değere neredeyse kesinlikle yakınsadığını belirtir.

\[ \frac{1}{n} \sum_{i=1}^{n} X_i \xrightarrow{a.s.} \mu \]

\begin{itemize}
	\item $X_i$: i'inci bağımsız ve özdeş dağılımlı rastgele değişken.
	\item $n$: Gözlem sayısı.
	\item $\mu$: Beklenen değer.
	\item $\xrightarrow{a.s.}$: Neredeyse kesinlikle yakınsama.
\end{itemize}




\newpage