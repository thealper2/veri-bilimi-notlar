\section{Paradigms}

\begin{itemize}
    \item \textbf{Instance-Based Learning:} Temelinde benzerlik ölçütüne dayalı olarak örnekler arasındaki benzerliklerin belirlenmesi ve bu benzerliklere dayalı tahminlerin yapılması vardır. Model öğrenme sürecinde örnek verileri doğrudan kullanır. Yani model, eğitim verilerindeki her bir örneği hatırlar ve bu örnekleri kullanarak yeni veri noktalarını tahmin eder. Örneğin KNN.
    \item \textbf{Model-Based Learning:} Bir modelin belirli bir görevi öğrenmek için bir eğitim süreci üzerinden genelleme yeteneğini kullanmasına dayanır. Örneğin lineer regresyon, destek vektör makineleri, karar ağaçları.
    \item \textbf{Interpretable Machine Learning:} Modellerin içerdikleri bilgi ve karar süreçlerinin insanlar tarafından anlaşılabilir ve yorumlanabilir olması anlamına gelir.
    \item \textbf{Applied Conformal Prediction:} Güvenilir tahminler yapma. Modelin öğrenme sürecindeki belirsizleri değerlendirmek ve bu belirsizlikleri tahmin sonuçlarına entegre etmek amacıyla tasarlanmıştır. Bu, modelin yaptığı tahminin güven aralığı ile birlikte sunulmasını sağlar. Model, yeni bir örneği tahmin ettiğinde, aynı örneği tekrar tekrar öğrenerek ve her seferinde farklı bir eğitim alt kümesini kullanarak bir güven aralığı oluşturur.
\end{itemize}

\newpage