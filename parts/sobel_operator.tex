\section{Sobel Operator}

Görüntüdeki kenarları tespit etmek için kullanılır. Gürültüye karşı dayanıklıdır. Hesaplama açısından verimlidir. Sobel operatörü, bir görüntüykide x ve y yönündeki gradyanları hesaplar. Bu işlem iki farklı 3x3 konvolüsyon çekirdeği ile gerçekleştirilir.

\[
G_x = \begin{bmatrix}
-1 & 0 & 1 \\
-2 & 0 & 2 \\
-1 & 0 & 1
\end{bmatrix}
\]

\[
G_y = \begin{bmatrix}
-1 & -2 & -1 \\
0 & 0 & 0 \\
1 & 2 & 1
\end{bmatrix}
\]

\subsection{Çalışma Adımları}
\begin{enumerate}
	\item Renkli görüntü gri tonlamalı görüntüye çevrilir.
	\item X ve Y yönündeki gradyanlar sobel çekirdekleri ile konvolüsyon işlemi uygulanarak hesaplanır.
	\item X ve Y yönündeki gradyanlar birleştirilerek her piksel için gradyan büyüklüğü hesaplanır.
	\item Gradyan büyüklükleri normalize edilir.
\end{enumerate}

\subsection{OpenCV - Sobel Operator}
\begin{lstlisting}[language=Python]
import cv2
import numpy as np
import matplotlib.pyplot as plt

image = cv2.imread('image.jpeg')
gray = cv2.cvtColor(image, cv2.COLOR_BGR2GRAY)
grad_x = cv2.Sobel(gray, cv2.CV_64F, 1, 0, ksize=3)
grad_y = cv2.Sobel(gray, cv2.CV_64F, 0, 1, ksize=3)
magnitude = cv2.magnitude(grad_x, grad_y)
normalized_magnitude = cv2.normalize(magnitude, None, 0, 255, cv2.NORM_MINMAX)
normalized_magnitude = np.uint8(normalized_magnitude)

plt.figure(figsize=(10, 5))
plt.subplot(1, 2, 1)
plt.imshow(gray, cmap='gray')
plt.title('Orijinal Gri Tonlamali Goruntu')
plt.axis('off')
plt.subplot(1, 2, 2)
plt.imshow(normalized_magnitude, cmap='gray')
plt.title('Sobel Kenar Algilama Sonucu')
plt.axis('off')
plt.show()
\end{lstlisting}

\newpage