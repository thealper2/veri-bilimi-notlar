\section{Barlow Twins}

2021 yılında Jure Zbontar ve arkadaşları tarafından önerilmiştir ve temelde, modelin aynı girdi verisinin farklı görünümlerinden (augmentations) benzer temsiller (representations) öğrenmesini amaçlar.

Barlow Twins’in temel çalışma prensibi, aynı girdi verisinin farklı dönüşümlerinden benzer temsiller öğrenmektir. Bu öğrenme süreci, veri örneklerinin kendiliğinden oluşturulmuş iki görünüşü arasında bir korelasyon matrisini optimize etmeye dayanır.

\begin{itemize}
    \item İlk olarak, aynı veri örneği üzerinde farklı veri iyileştirmeleri (augmentations) yapılır.
    \item Her iki farklı görünüm (augmentation), aynı ağdan (ikiz mimari) geçirilir.
    \item Model, bu iki görünüm arasındaki temsillerin benzer olmasını sağlamaya çalışır. Barlow Twins'in temel yeniliği, temsiller arasındaki korelasyon matrisinin ideal hale getirilmesi üzerine kuruludur.
\end{itemize}

Barlow Twins, her iki görünümün çıktı temsilleri arasında bir korelasyon matrisi oluşturur ve bu matrisin birim matrise yakın olmasını hedefler. Bunun için şu iki kritere bakar:

\begin{itemize}
    \item \textbf{Self-correlations}: Aynı özellikler arasında (aynı verinin iki görünümünden gelen aynı bileşenler) yüksek korelasyon sağlanmalı.
    \item \textbf{Cross-correlations}: Farklı özellikler arasında (aynı verinin iki görünümünden gelen farklı bileşenler) minimum korelasyon olmalı.
\end{itemize}

\newpage