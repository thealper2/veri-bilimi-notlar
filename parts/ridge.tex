\section{Ridge}
Çok değişkenli regresyon verilerini analiz etmek için kullanılır. Lineer regresyonun bir türevidir ve genellikle aşırı uydurma (overfitting) problemini çözmek için kullanılır. Model, lineer regresyonun temel yapısını korurken, katsayıları sınırlamak için L2 düzenlemesi (penalty) ekler. Bu düzenleme, kaysayıların büyüklüğünü sınırlayarak modeli daha da genelleştirilmiş hale getirir. Tüm değişkenler ile modeli oluşturur, ilgisiz değişkenleri çıkarmaz fakat katsayılarını sıfıra yaklaştırır.

\subsection{Hiperparametreler}
\begin{table}[h]
\centering
{\scriptsize\renewcommand{\arraystretch}{0.4}
{\resizebox*{\linewidth}{0.4\textwidth}{
\begin{tabular}{|p{3cm}|p{1cm}|p{1cm}|p{6cm}|}
\hline
Parametre & Type & Default & Açıklama \\ \hline
alpha & float & 1.0 & L2 düzenlemesi katsayısıdır. Bu parametre arttıkça katsayıların büyüklüğüne daha fazla düzenleme uygulanır. \\ \hline
fit\_intercept & bool & True & Kesişimin (intercept) uydurulup uydurulmayacağı. \\ \hline
copy\_X & bool & True & Eğer True ise modeli eğitirken X değeri fonksiyonda kullanılacak ve eğitimden sonra da aynı olacaktır. False olduğunda ise X fonksiyona girdikten sonra ilk hali ile aynı olmayabilir. \\ \hline
max\_iter & int & None & Gradyan çözücüsü için maximum iterasyon sayısı. \\ \hline
tol & float & 1e-4 & Katsayı hassasiyeti (coef) \\ \hline
solver & tol & 1e-4 & Hesaplamada kullanılacak çözücü. \\ \hline
positive & string & "auto" & True ise katsayıları pozitif olmaya zorlar. \\ \hline
random\_state & bool & False & Seed \\ \hline
\end{tabular}
}}}
\end{table}

\newpage