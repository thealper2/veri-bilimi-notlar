\section{Individual Explained Variance ve Cumulative Explained Variance}
Bireysel Açıklanan Varyans (İndividual Explained Variance), her bir bileşenin veya boyut indirgeme işlemi sonucunda elde edilen her bir yeni değişkenin, toplam varyansın ne kadarını açıkladığını belirtir. Her bir bileşen veya değişken için ayrı ayrı hesaplanır ve bu bileşenin orijinal veri setinin varyansının ne kadarını açıkladığını gösterir. Her bileşen veya değişkenin açıkladığı varyans, bu bileşenin veya değişkenin önemini belirlemede kullanılır.

Kümülatif Açıklanan Varyans (Cumulative Explained Variance), boyut indirgeme işlemi sırasında elde edilen her bir bileşen veya yeni değişkenin, toplam varyansın ne kadarını açıkladığını biriktirir. Yani, her bir bileşenin veya değişkenin açıkladığı varyansların toplamıdır. Kümülatif açıklanan varyans, boyut indirgeme işleminin ardından ne kadar varyansın korunduğunu ve orijinal veri setinin ne kadarının korunduğunu değerlendirmek için kullanılır. Kümülatif açıklanan varyans grafiği, boyut indirgeme işleminin ardından ne kadar bilginin korunduğunu görselleştirir.

Örneğin bir faktör analizinde 3 faktör olduğunu varsayalım. Faktör 1 \%40 varyans, faktör 2 \%20 varyans ve faktör 3 \%10 varyans açıklıyor. Kümülatif açıklanan varyans; 
\begin{itemize}
	\item \textbf{Faktör 1:} \%40
	\item \textbf{Faktör 2:} \%60 (40 + 20)
	\item \textbf{Faktör 3:} \%70 (60 + 10)
\end{itemize}

Bireysel açıklanan varyans;
\begin{itemize}
	\item \textbf{Faktör 1:} \%40
	\item \textbf{Faktör 2:} \%20
	\item \textbf{Faktör 3:} \%10
\end{itemize}

\newpage