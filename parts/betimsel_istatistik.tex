\section{Betimsel İstatistik}
Betimsel istatistik, veri kümelerini özetlemek ve analiz etmek için kullanılan bir dizi yöntemdir. İki gruba ayrılır:
\begin{enumerate}
	\item \textbf{Merkezi Eğilim Ölçüleri}: Ortalama, Mod, Medyan.
	\item \textbf{Yayılım Ölçüleri}: Standart Sapma, Varyans, Minimum ve Maksimum Değerler.
\end{enumerate}

\newpage

\subsection{Toplam (Sum) ve Ortalama (Mean)}
\subsubsection{Toplam (Sum)}

\[
\sum_{i=1}^{n} x_i
\]

\subsubsection{R Kodu}

\begin{lstlisting}[language=R]
x <- c(12, 10, 14)
sum(x) # 36
\end{lstlisting}

\newpage

\subsubsection{Ortalama (Mean)}

\[
\bar{x} = \frac{1}{n} \sum_{i=1}^{n} x_i
\]

\subsubsection{R Kodu}

\begin{lstlisting}[language=R]
x <- c(12, 10, 14)
ort <- sum(x) / length(x)
ort # 12
mean(x) # 12
\end{lstlisting}

\newpage

\subsection{Serbestlik Derecesi}
Serbestlik derecesi (degrees of freedom), bir veri kümesinin bağımsız bilgi parçalarının sayısını ifade eder. ANOVA, ki-kare, t-testinde kullanılır. Bu testlerde testin anlamlılık düzeyini ve kritik değerlerini belirlemede kullanılır. Örneğin 7 adet şapka var ve haftanın her bir günü bu şapkalardan biri giyilecek. Pazar günü sadece 1 şapka seçimi kalır ve bu seçimde bir serbestlik olmaz.

\[
\text{Serbestlik Derecesi} = \text{Gozlem Sayisi} - 1
\]

\newpage

\subsection{Standart Sapma (Standard Deviation)}
Standart sapma (Standard Deviation), bir veri kümesinin ortalama etrafındaki dağılımını ölçen bir istatistiksel değerdir. Verilerin merkezi eğilimden ne kadar saptığını gösterir. Düşük standart sapma, verilerin ortalama etrafında yoğunlaştığını; yüksek standart sapma verilerin geniş bir alana yayıldığını ve değerlerin birbirine uzak olduğunu gösterir.

\subsubsection{R Kodu}

\begin{lstlisting}[language=R]
x <- c(12, 10, 14)
sd(x) # 2
\end{lstlisting}

\subsubsection{Ortalaması Bilinen Bir Kümenin Standart Sapması}

\[
\sigma = \sqrt{\frac{1}{N} \sum{i=1}^{N} (x_i - \mu)^2}
\]

\begin{itemize}
    \item $N$: Veri kümesindeki toplam gözlem sayısı.
    \item $x_i$: i'inci gözlem.
    \item $\mu$: Veri kümesinin ortalaması
\end{itemize}

\subsubsection{Ortalaması Bilinmeyen Bir Kümenin Standart Sapması}

\[
\sigma = \sqrt{\frac{1}{n-1} \sum_{i=1}^{n} (x_i - \bar{x})^2}
\]

\begin{itemize}
    \item $n$: Örneklemdeki toplam gözlem sayısı.
    \item $x_i$: i'inci gözlem.
    \item $\bar{x}$: Örneklem ortalaması. 
\end{itemize}

\newpage

\subsection{Varyans (Variance)}
Varyans, standart sapmanın karesidir. Standart sapmadan farkı, standart sapma veri kümesi içindeki gözlemlerin ortalamadan farklılığını tanımlar, varyans ise değişkenliğini tanımlanır. Varyans, bir veri kümesinin ortalama etrafındaki dağılımını ölçer. Yani her bir gözlemin ortalamadan farkının karesinin ortalaması olarak tanımlanır. Verilerin ne kadar geniş bir aralığa yayıldığını veya ne kadar değişken olduğunu gösterir. Düşük varyans verilerin ortalama etrafında yoğunlaştığını; yüksek varyans verilerin geniş bir alana yayıldığını gösterir. 

\[
\sigma^2 = \frac{1}{N} \sum_{i=1}^{N} (x_i - \mu)^2
\]

\begin{itemize}
    \item $n$: Veri kümesindeki toplam gözlem sayısı.
    \item $x_i$: i'inci gözlem.
    \item $\mu$: Veri kümesinin ortalaması. 
\end{itemize}

\subsubsection{R Kodu}

\begin{lstlisting}[language=R]
x <- c(12, 10, 14)
var(x) # 4
\end{lstlisting}

\newpage

\subsection{Medyan (Median)}
Bir veri kümesi sıralandığında ortada kalan medyan değeridir. Veri kümesi çift sayıda gözlem içeriyorsa, medyan, ortadaki iki değerin aritmedik ortalamasıdır. Medyan, veri kümesinin merkezi eğilim ölçüsü olarak kullanılır. Uç değerlerin etkisini azaltmak için yararlıdır.

\subsubsection{R Kodu}

\begin{lstlisting}[language=R]
x <- c(12, 10, 14)
median(x) # 12
\end{lstlisting}

\subsubsection{Tek Sayıdaki Gözlem için Medyan}

\[
\text{Median} = x_{\left(\frac{n+1}{2}\right)}
\]

\subsubsection{Çift Sayıdaki Gözlem için Medyan}

\[
\text{Median} = \frac{x_{\left(\frac{n}{2}\right)} + x_{\left(\frac{n}{2}+1\right)}}{2}
\]

\newpage

\subsection{Açıklık (Range)}
Açıklık, veri kümesindeki en yüksek ve en düşük değer arasındaki farktır. Bu, veri kümesinin yayılımını gösterir.

\[
\text{Range} = \max(x_i) - \min(x_i)
\]

\subsubsection{R Kodu}

\begin{lstlisting}[language=R]
x <- c(12, 10, 14)
range(x) # 10 - 14
\end{lstlisting}

\newpage

\subsection{Çeyreklikler (Quartiles)}
Çeyreklikler, bir veri kümesini 4 eşit parçaya bölen değerlerdir. Üç ana çeyreklik vardır; birinci çeyreklik (Q1), ikinci çeyreklik (Q2) veya medyan (medyan ile aynıdır) ve üçüncü çeyreklik (Q3). Veri setinin dağılımını anlamak için kullanılır.

\[
Q1 = x_{\left(\frac{n+1}{4}\right)}
Q2 = x_{\left(\frac{n+1}{2}\right)}
Q3 = x_{\left(\frac{3(n+1)}{4}\right)}
\]

\subsubsection{R Kodu}

\begin{lstlisting}[language=R]
x <- c(23, 53, 95, 12, 45)
quantile(x)
# 0% = 12
# 25% = 23
# 50% = 45 (medyan)
# 75% = 53
# 100% = 95
\end{lstlisting}

\newpage

\subsection{Mod (Mode)}
Veri kümesi içinde en sık görülen değerdir. Tepedeğer olarak da bilinir.

\newpage

\subsection{Çarpıklık Katsayısı (Skewness Coefficient)}
Çarpıklık katsayısı, bir veri kümesinin simetrisini ölçen bir değerdir. Çarpıklık, veri dağılımının ortalama etrafında nasıl şekillendiğini gösterir. Çarpıklık katsayısı genelde -1 ile 1 arasında değer alır. 0-1 arasında ise sağa çarpık, -1-0 arasında ise sola çarpıktır. 
\begin{itemize}
	\item \textbf{Pozitif Çarpıklık}: Sağ çarpık dağılım. Veri kümesindeki verilerin çoğu düşük değerdedir.
	\item \textbf{Negatif Çarpıklık}: Sol çarpık dağılım. Veri kümesindeki verilerin çoğu yüksek değerdedir.
	\item \textbf{Sıfır Çarpıklık}: Simetrik dağılım, normal dağılım.
\end{itemize}


\[
\text{Skewness} = \frac{n}{(n-1)(n-2)} \sum_{i=1}^{n} \left( \frac{x_i - \bar{x}}{s} \right)^3
\]

\begin{itemize}
	\item $n$: Veri kümesindeki gözlem sayısı.
	\item $x_i$: i'inci gözlem.
	\item $\bar{x}$: Veri kümesinin ortalaması
	\item $s$: Veri kümesinin standart sapması.
\end{itemize}

\subsubsection{R Kodu}

\begin{lstlisting}[language=R]
library('e1071')

x <- c(23, 53, 95, 12, 45)
skewness(x) # 0.43010, pozitif (sag) carpik
\end{lstlisting}

\newpage

\subsection{Frekans (Frequency)}
Sıklık olarak da bilinir. Veri kümesindeki bir değişkenin tekrar sayısına frekans denir.

\subsubsection{R Kodu}

\begin{lstlisting}[language=R]
x <- c(12, 23, 23, 23, 53, 95, 12, 45)
table(x)
# 12 23 45 53 95 
#  2  3  1  1  1
\end{lstlisting}

\newpage