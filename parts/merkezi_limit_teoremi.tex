\section{Merkezi Limit Teoremi}
Örneklem dağılımının şekli ne olursa olsun, bir popülasyondan alınan bağımsız ve özdeş dağılımlı örneklemlerin ortalamalarının dağılımının, örneklem büyüklüğü yeterince büyük olduğunda, yaklaşık olarak normal dağılıma yakınsadığını ifade eder. 

Bir popülasyondan çekilen $n$ büyüklüğündeki bağımsız ve özdeş dağılımlı rastgele değişkenlerin $X_1, X_2, ..., X_n$ ortalaması $\bar{X}$ şu şekildedir;

\[ \bar{X} = \frac{1}{n} \sum_{i=1}^{n} X_i \]

Bu rastgele değişkenlerin beklenen değeri $\mu$ ve varyansı $\sigma^2$ olsun. Merkezi Limit Teoremi'ne göre $n$ yeterince büyük olduğunda $\bar{X}$'in dağılımı yaklaşık olarak şu şekilde normal dağılıma yaklaşır;

\[ \bar{X} \sim \mathcal{N}\left(\mu, \frac{\sigma^2}{n}\right) \]

Daha genel bir ifade ile, normalize edilmiş toplamların dağılımı;

\[ \frac{\sum_{i=1}^{n} X_i - n\mu}{\sigma\sqrt{n}} \xrightarrow{d} \mathcal{N}(0, 1) \]

Burada $\xrightarrow{d}$ ifadesi dağılımsal olarak yakınsar anlamına gelir ve $\mathcal{N}(0, 1)$ standart normal dağılımı ifade eder.

\newpage