\section{DenseNet}
DenseNet (Densely Connected Convolutional Networks), 2016 yılında Gao Huang, Zhuang Liu, ve Kilian Q. Weinberger tarafından tanıtmıştır. DenseNet, her katmandaki çıkışın, bir önceki katmandaki çıkışlarla birleştirildiği yoğun bağlantılar kurar. Yani, her katmandaki nöronlar, kendilerinden önceki katmandaki nöronların çıktılarına erişebilir. Bu yoğun bağlantılar, ağın daha önceki katmanlardan gelen bilgiyi daha iyi kullanmasına ve gradyanın daha etkili bir şekilde akmasına sağlar. Ayrıca, bu yapı ağın daha derin olmasını ve daha fazla parametreye sahip olmasını mümkün kılar. DenseNet, genellikle çok sayıda katman içerir ve daha fazla derinlik eklemek, daha karmaşık desenleri ve özellikleri tanımayı sağlar. Yoğun bağlantılar, daha fazla parametre eklemeye rağmen, hesaplama verimliliğini artırır ve daha az eğitim verisi gerektirir. Ağın derinliğini artırırken boyutu azaltmak için özel azaltma katmanları kullanılır.

\newpage