\section{Multiple Instance Learning (MIL)}

Etiketlenmiş verilerin tam olarak bilinmediği durumlarda kullanılır. MIL'de veriler "çanta" (bag) adı verilen gruplar halinde sunulur. Her çanta, birden fazla örnek (instance) içerir. Bu çantaların tümüne bir etiket atanır ancak bireysel örneklerin etiketleri bilinmez. Etiketleme seviyesi çanta bazında yapılır yani bir çantadaki örneklerin tamamı üzerinde bir sınıflandırma gerçekleştirilir. 

\begin{itemize}
    \item \textbf{Pozitif Çanta}: Eğer bir çantadaki herhangi bir örnek pozitif sınıfa ait ise çanta pozitif olarak etiketlenir.
    \item \textbf{Negatif Çanta}: Eğer bir çantadaki herhangi bir örnek negatif sınıfa ait ise çanta negatif olarak etiketlenir.
\end{itemize}

\newpage