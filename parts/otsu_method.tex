\section{OTSU Method}
1979 yılında Nobuyuki Otsu tarafından geliştirilmiştir. Bir görüntüyü ikili hale getirmek içi kullanılır. Görüntüyü iki sınıfa ayırarak bu sınıflar arasındaki piksel yoğunluk dağılımını maksimize eden eşik değerini otomatik olarak belirler. 

\subsection{Çalışma Adımları}
\begin{enumerate}
	\item Görüntünün piksel yoğunluk histogramı hesaplanır.
	\item Her eşik değeri için sınıf olasılıkları ve ortalama yoğunluklar hesaplanır.
	\item Her olası eşik değeri için sınıflar arası varyans hesaplanır.
	\item Sınıflar arası varyansı maksimize eden eşik değeri belirlenir.
	\item Görüntü bu eşik değerine göre ikili hale getirilir.
\end{enumerate}

\subsection{OpenCV - OTSU Method}
\begin{lstlisting}[language=Python]
import cv2
import numpy as np
import matplotlib.pyplot as plt

image = cv2.imread('image.jpg', 0)
blur = cv2.GaussianBlur(image, (5, 5), 0)
_, otsu_threshold = cv2.threshold(blur, 0, 255, cv2.THRESH_BINARY + cv2.THRESH_OTSU)

plt.figure(figsize=(10, 5))
plt.subplot(1, 2, 1)
plt.imshow(image, cmap='gray')
plt.title('Orijinal Gornntu')
plt.axis('off')
plt.subplot(1, 2, 2)
plt.imshow(otsu_threshold, cmap='gray')
plt.title('Otsu Esikleme Sonucu')
plt.axis('off')
plt.show()
\end{lstlisting}

\newpage