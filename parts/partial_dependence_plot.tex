\section{Partial Dependence Plot (PDP)}
Değişkenlerin öngörüler üzerindeki etkilerini görselleştirmektir. Belirli bir bağımsız değişkenin diğer değişkenlere göre bağımlığını gösterir. PDP tüm değişkenleri bağımsız sayar. Bir değişkenin diğer değişkenlerle etkileşimlerini gösterir.

\subsection{Adımlar}
\begin{itemize}
    \item Değişken Seçimi: İlgilenilen bağımsız değişken seçilir.
    \item Değişkenin Değer Aralığının Belirlenmesi: Seçilen değişkenin olası değer aralığı belirlenir.
    \item Diğer Değişkenlerin Değerleri Sabit Tutma: Diğer bağımsız değişkenlerin değerleri sabit tutularak, seçilen değişkenin etkisi incelenir.
    \item Modelin Tahminlerinin Alınması: Değişkenin değeri değiştirilerek, model üzerinden tahminler alınır.
    \item Sonuçların Görselleştirilmesi
\end{itemize}

\subsection{Python Kodu}

\begin{lstlisting}[language=Python]
import numpy as np
import matplotlib.pyplot as plt
from sklearn.datasets import load_boston
from sklearn.inspection import plot_partial_dependence
from sklearn.ensemble import GradientBoostingRegressor

boston = load_boston()
X, y = boston.data, boston.target

model = GradientBoostingRegressor(n_estimators=50, max_depth=6, learning_rate=0.1, loss='huber', random_state=4242)

model.fit(X, y)

features = [5]
plot_partial_dependence(model, X, features, grid_resolution=50, feature_names=boston.feature_names)

plt.show()
\end{lstlisting}

\newpage