\section{Modelleme Süreçleri}

\subsubsection{CRISP-DM} 

İş problemini anlama ve veri bilimiyle nasıl çözüleceğini belirler. Veriyi keşfetme ve iş problemini çözme potansiyelini anlama. 	Veri temizleme, dönüştürme, ve modelleme için uygun hale getirme. Uygun veri madenciliği modellerini oluşturma ve test etme. Model sonuçlarını iş problemleriyle uyumlu olup olmadığını değerlendirme. Modelin gerçek dünyada nasıl kullanılacağını planlama ve uygulama.

\subsubsection{SEMMA}

İş problemi vurgulanmaz, daha çok verinin analizi ön plandadır. Veriyi keşfetmek, ilişkileri ve kalıpları incelemek. Veriyi dönüştürme ve değiştirme, modellemeye hazırlık. Modelleme adımı çok güçlüdür, farklı algoritmalar uygulanır. Modelin başarısını ve uygunluğunu değerlendirme. Sonuçları uygulama adımı doğrudan yer almaz.

\subsubsection{KDD}

İş problemi açıkça yer almaz, bilgi keşfi amaçlanır. Veriyi seçmek ve analiz için ilgili veriyi ayıklamak. Veri temizleme, eksik veri yönetimi, dönüştürme. Veri madenciliği teknikleriyle bilgi çıkarma. Keşfedilen bilgilerin anlamlı ve kullanışlı olup olmadığını değerlendirme. Model sonuçlarının iş süreçlerine entegrasyonu odaklanmaz.

\newpage

\subsection{CRISP-DM (Cross Industry Standard Process for Data Mining)}

CRISP-DM, 1996 yılında bir grup veri bilimci tarafından geliştirilmiştir. CRISP-DM’in temel amacı, iş problemlerini veriye dayalı bir şekilde çözmek için yapısal ve tekrarlanabilir bir süreç sunmaktır. Veri madenciliği ve veri bilim projelerinde karmaşık süreçleri yönetirken, veri bilimcilerinin ve iş liderlerinin ortak bir dilde konuşmasını sağlar. Ayrıca, projelerde hangi adımların atılacağını ve hangi sonuçların beklenmesi gerektiğini açıkça belirlediği için projenin başından sonuna kadar bir çerçeve sunar. Bu, projelerin başarısını artırır, riskleri azaltır ve daha verimli sonuçlar elde edilmesine yardımcı olur.

\begin{enumerate}
    \item \textbf{İş Anlayışı (Business Understanding)}: Proje için iş hedeflerini ve beklentileri anlamaktır. Projeye başlamadan önce, iş problemi tanımlanır ve hangi sorunun çözüleceği netleştirilir.
    \item \textbf{Veri Anlayışı (Data Understanding)}: Veriyi keşfetmek ve analiz etmektir. Veri seti toplanır, veri kalitesi kontrol edilir ve veri yapısı ile ilgili ilk analizler yapılır.
    \item \textbf{Veri Hazırlığı (Data Prepartion)}: Veriyi modelleme için hazır hale getirmektir. Verideki eksikliklerin doldurulması, gereksiz değişkenlerin çıkarılması ve gerekli veri dönüşümleri bu adımda yapılır.
    \item \textbf{Modelleme (Modelling)}: Seçilen algoritmalarla veri üzerinde modelleme yapmaktır. Bu aşamada farklı modelleme teknikleri kullanılarak, probleme uygun en iyi model geliştirilir.
    \item \textbf{Değerlendirme (Evaluation)}: Oluşturulan modelin, iş hedefleriyle ve veri madenciliği hedefleriyle ne kadar örtüştüğünü değerlendirmektir. Modelin iş problemini gerçekten çözüp çözmediği ve performansının yeterli olup olmadığı analiz edilir.
    \item \textbf{Uygulama (Deployment)}: Modeli iş süreçlerine entegre etmektir. Geliştirilen modelin karar verme süreçlerine katkı sağlaması için iş operasyonlarına entegre edilmesi gerekir.
\end{enumerate}

\newpage

\subsection{SEMMA (Sample, Explore, Modify, Model, Assess)}

SEMMA, SAS Institute tarafından geliştirilmiştir. SEMMA, adımların baş harflerinden oluşan bir kısaltmadır ve veri bilim projelerinde veriyi anlamak, temizlemek, modellemek ve sonuçları değerlendirmek için yapısal bir süreç sunar. CRISP-DM gibi geniş kapsamlı iş problemlerine odaklanmak yerine daha çok veri hazırlama ve modelleme süreçlerine odaklanır. SEMMA'nın temel amacı, veriden anlamlı modeller ve sonuçlar çıkarabilmek için adım adım bir süreç sunmaktır.

\begin{enumerate}
    \item \textbf{Örnekleme (Sample)}: Büyük veri setinden bir alt küme seçmek ve analiz için kullanmaktır.
    \item \textbf{Keşfetme (Explore)}: Veriyi keşfetmek, analiz etmek ve veri setinin temel özelliklerini anlamaktır. Bu adımda, veri yapısındaki dağılımlar, ilişkiler, eksiklikler ve anomaliler tespit edilir.
    \item \textbf{Değiştirme (Modify)}: Veriyi modellemeye uygun hale getirmek için dönüştürme ve temizleme işlemlerini yapmaktır. Bu adımda özellik mühendisliği, eksik verilerin doldurulması, yeni değişkenlerin türetilmesi gibi işlemler yapılır.
    \item \textbf{Modelleme (Model)}: Hazırlanan veri üzerinde farklı modelleme tekniklerini kullanarak tahmin modelleri oluşturmak ve sonuçlarını değerlendirmektir. Bu adımda istatistiksel modeller, makine öğrenimi algoritmaları ve diğer veri madenciliği teknikleri kullanılır.
    \item \textbf{Değerlendirme (Assess)}: Oluşturulan modellerin performansını ve doğruluğunu değerlendirmektir. Bu adımda, modellerin gerçek hayattaki performansı simüle edilir ve modelin iş problemini ne kadar iyi çözdüğü analiz edilir.
\end{enumerate}

\newpage

\subsection{KDD (Knowledge Discovery in Databases)}

Türkçede "Veritabanlarında Bilgi Keşfi" olarak bilinir. 1990'larda geliştirilmiştir. KDD, ham veriyle başlayıp anlamlı bilgiyi ortaya çıkaran bir süreçtir.

\begin{enumerate}
    \item \textbf{Veri Seçimi (Data Selection)}: Büyük veri setinden analiz için uygun verilerin seçilmesidir. Tüm veriyi kullanmak yerine, analiz için daha anlamlı ve ilgili verilerin seçilmesi sürecin ilk adımıdır.
    \item \textbf{Veri Ön İşleme (Preprocessing)}: Verinin modellemeye hazır hale getirilmesi için temizlenmesi ve dönüştürülmesidir. Ham veri genellikle hatalar, eksiklikler veya gürültüler içerir. Bu adımda, veri eksiklikleri tamamlanır, hatalar düzeltilir ve gürültüden arındırılır.
    \item \textbf{Veri Dönüştürme (Data Transformation)}: Verinin analiz ve modelleme için daha uygun bir forma dönüştürülmesidir. Özellik çıkarımı, veri özetleme, boyut indirgeme gibi işlemler bu aşamada gerçekleştirilir.
    \item \textbf{Veri Madenciliği (Data Mining)}: Verinin içindeki gizli kalıpların, ilişkilerin ve trendlerin keşfedilmesidir. Bu adım, makine öğrenimi ve istatistiksel analiz tekniklerini kullanarak veriden anlamlı bilgi elde etmeye odaklanır
    \item \textbf{Sonuçların Yorumlanması ve Değerlendirilmesi (Interpretation and Evaluation)}: Veri madenciliği sonucunda elde edilen modellerin ve kalıpların değerlendirilmesidir. Bu adımda, keşfedilen bilginin iş problemi açısından anlamlı ve kullanılabilir olup olmadığı analiz edilir.
\end{enumerate}

\newpage