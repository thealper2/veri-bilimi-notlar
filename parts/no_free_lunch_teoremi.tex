\section{No Free Lunch Teoremi}
NFL teoremi 1997 yılıdna David Wolpert ve William Macready tarafından ortaya atılmıştır. Temel fikir, herhangi bir algoritmanın genel olarak en iyi performansı sağlayacak bir formülasyona sahip olmadığıdır. Bu nedenle, bir algoritmanın başarısının, belirli bir problem bağlamında nasıl ölçüldüğüne ve problem setinin doğasına bağlı olduğunu belirtir. Optimizasyon problemlerinde her algoritmanın eşit derecede başarılı olduğunu ve hiçbir algoritmanın her problemde en iyi performansı göstermeyeceğini savunur. 
Teorem, optimizasyon problemlerini iki kategoriye ayırır:
\begin{itemize}
	\item \textbf{Problem Sınıfı:} Tüm olası problemlerden oluşan bir kümedir.
	\item \textbf{Algoritma Sınıfı:} Tüm olası optimizasyon algoritmalarından oluşan bir kümedir.
\end{itemize}

Teoreme göre, herhangi bir algoritma sınıfı için, problem sınıfının her alt kümesi için, algoritma sınıfındaki algoritmaların ortalama performansı aynıdır. Yani, bir algoritma alt kümede diğerlerinden daha iyi performans gösteriyorsa, bu performans diğer alt kümelerde daha düşük olacaktır.

NFL teoremi, herhangi bir makine öğrenimi algoritmasının tüm problemler için en iyi performans göstermediğini belirtir. Yani, bir algoritmanın belirli bir problemde iyi performans göstermesi, diğer problemlerde de aynı performansı göstereceği anlamına gelmez. 

\newpage