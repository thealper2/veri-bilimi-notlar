\section{Non-negative Matrix Factorization (NMF)}
Non-negative Matrix Factorization (NMF), çok değişkenli verilerde kullanılan bir matris faktörizasyon tekniğidir. NMF, bir veri matrisini iki veya daha fazla alt matrise çözer. Bu faktörizasyon, orijinal matrisi daha küçük boyutlu ve daha yorumlanabilir alt matrislere bölerek veri içindeki gizli yapıları keşfetmeyi amaçlar. Temel amacı, veri matrisindeki her bir özelliği temsil eden bazı temel özelliklerin kombinasyonunu bulmaktır.

\begin{lstlisting}[language=Python]
from sklearn.decomposition import NMF
import numpy as np

# Ornek veri olusturma
X = np.array([[1, 1, 1], [2, 2, 2], [3, 3, 3], [4, 4, 4]])

# NMF modelini tanimlama ve egitme
num_components = 2  # Alt matrislerin sayisi
nmf_model = NMF(n_components=num_components, init='random', random_state=42)
W = nmf_model.fit_transform(X)  # Birinci alt matris
H = nmf_model.components_  # Ikinci alt matris

# Olusturulan alt matrislerin gorsellestirilmesi
print("W matrix:")
print(W)
print("H matrix:")
print(H)
\end{lstlisting}

\newpage