\section{Point-E}
OpenAI tarafından geliştirilmiştir. Metinden 3D modeller oluşturmayı sağlar. İlk olarak verilen metinden görüntü modeli kullanılarak difüzyon ile bir görüntü oluşturulur. Daha sonra bu görseli kullanarak 3D model ile nesneler oluşturulur. 

\subsection{Metinden 3D Nesne Oluşturma}
\begin{lstlisting}[language=Python]
import torch
import plotly.graph_objects as go

from point_e.models.download import load_checkpoint
from point_e.diffusion.sampler import PointCloudSampler
from point_e.diffusion.configs import DIFFUSION_CONFIGS , diffusion_from_config
from point_e.models.configs import MODEL_CONFIGS , model_from_config

device = torch.device("cuda" if torch.cuda.is_available() else "cpu")

base_model = model_from_config(MODEL_CONFIGS["base40M-textvec"] , device)
base_model.eval()
base_diffusion = diffusion_from_config(DIFFUSION_CONFIGS["base40M-textvec"])

upsampler_model = model_from_config(MODEL_CONFIGS["upsample"], device)
upsampler_model.eval()
upsampler_diffusion = diffusion_from_config(DIFFUSION_CONFIGS["upsample"])

base_model.load_state_dict(load_checkpoint("base40M-textvec" , device))
upsampler_model.load_state_dict(load_checkpoint("upsample" , device))

sampler = PointCloudSampler(
    device = device,
    models = [base_model , upsampler_model],
    diffusions = [base_diffusion , upsampler_diffusion],
    num_points = [1024 , 4096 - 1024],
    aux_channels = ["R" , "G" , "B"],
    guidance_scale = [3.0, 0.0],
    model_kwargs_key_filter=("texts" , "")
)

prompt = 'a BLUE truck'
batch_size = 1
samples = None
for x in sampler.sample_batch_progressive(batch_size=batch_size, 
                                          model_kwargs=dict(texts=[prompt])):
    samples = x
    
pred = sampler.output_to_point_clouds(samples)[0]
rgb_values = zip(pred.channels["R"], pred.channels["G"], pred.channels["B"])
color = list(map(lambda rgb: "rgb({}, {}, {})".format(*rgb), rgb_values))

fig = go.Figure(
        data = [
            go.Scatter3d(
            x = pred.coords[:, 0], 
            y = pred.coords[:, 1], 
            z = pred.coords[:, 2],
            mode = "markers" ,
            marker = dict(
                    size = 2,
                    color = color,
                )
            )
        ],
        layout = dict(
            scene=dict(
                xaxis = dict(visible = False),
                yaxis = dict(visible = False),
                zaxis = dict(visible = False)
            )
        ),
)
fig.show()
\end{lstlisting}



\newpage