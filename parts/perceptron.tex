\section{Perceptron}
1950'lerde F. Rozenblat tarafından önerilmiştir. Yapay sinir ağlarının ilk modellerindendir. Perceptronların öğrenme algoritması, gelişmiş öğrenme algoritmalarının temelini oluşturmuştur. Perceptron, bir veya daha fazla nörona sahip tek bir katmandan oluşan en basit sinir ağı modelidir. Perceptron, birçok girişten oluşan bir vektörü alır ve bu girişlerin her birini belirli bir ağırlıkla çarparak bir çıktı üretir. Bu çıktı daha sonra bir eşik değeriyle karşılaştırılır ve bir aktivasyon fonksiyonu kullanılarak sonuç belirlenir. Ağırlıklar, başlangıçta rastgele atanmış bir diziyle başlar. Eğitim sürecinde, ağırlıkları güncellemek için geri yayılım algoritması kullanılır. Perceptron'un MLP'den farkı, Perceptron tek katmandan oluşur. MLP'ler ise birbirine bağı birçok perceptron katmanı içerir. Percetron'lar sadece ikili sınıflandırma problemlerinde kullanılırken MLP'ler ise birden fazla sınıflı sınıflandırma problemlerinde de kullanılır. Perceptron kullanılarak OR, AND gibi tek karar düzeyi gerektiren problemler çözülebilir.

\newpage