\section{ConvMixer}

CNN ve ViT mimarilerinin bazı fikirlerinin değiştirerek kullanır. ConvMixer, saf konvolüsyon işlemleriyle çok katmanlı bir yapıda global bilgi akışını sağlar ve ViT'deki "patch embedding" ile benzer bir strateji uygular.

\subsection{Çalışma Adımları}

\begin{enumerate}
    \item Girdi olarak verilen görüntü, sabit boyutlu yamalara (patch) ayrılır. Bu yamalar, ConvMixer'ın giriş katmanına aktarılır.
    \item Her yama üzerinde derinlemesine konvolüsyon işlemi yapılır. Bu adımda, her kanal birbirinden bağımsız olarak işlenir ve görüntünün uzamsal özellikleri çıkarılır.
    \item Her bir yama üzerinde kanal bazında konvolüsyon işlemi yapılır ve farklı kanalların bilgileri birleştirilir. Bu aşama, hem yerel hem de küresel ilişkilerin modellenmesini sağlar.
    \item Her konvolüsyon işleminin ardından aktivasyon fonksiyonları uygulanır ve artık bağlantılar ile öğrenme süreci hızlandırılır.
    \item Tüm yamalar ve kanallar üzerinde öğrenilen öznitelikler, bir global havuzlama katmanına aktarılır. Bu adım, özniteliklerin boyutunu küçültür ve sınıflandırma katmanına hazır hale getirir.
    \item Tam bağlantılı katman, havuzlanan öznitelikleri kullanarak nesnenin hangi sınıfa ait olduğunu tahmin eder. Bu aşamada model, son katmanında "softmax" fonksiyonunu kullanarak sınıflandırma işlemini tamamlar.
\end{enumerate}

\newpage