\section{Extra Trees}
Extremely Randomized Trees, ağaçların oluşturulması sırasında rastgelelik kullanarak bir dizi karar ağacı oluşturur ve bu ağaçların tahminlerini bir araya getirerek sonuçları üretir. Diğer ağaç tabanlı yöntemlerden farkı, ağaçların oluşturulması ve eğitim sürecindeki rastgelelik derecesinin daha yüksek olmasıdır.

\subsection{Çalışma Adımları}
\begin{itemize}
    \item Her bir ağaç oluşturulurken, özelliklerden rastgele bir alt küme seçilir.
    \item Her özellik için rastgele bir eşik değeri seçilir.
    \item Rastgele özellik ve eşik değerleri kullanılarak karar ağaçları oluşturulur. Maksimum derinliğe ulaşana kadar büyütülür.
    \item Bootstrap ile her ağacın farklı bir eğitim veri setiyle eğitilmesi sağlanır.
    \item Tüm ağaçlar üzerinde tahminler yapılır ve bu tahminler bir araya getirilerek modelin tahminleri üretilir. Sınıflandırma için en sık görülen tahmin alınır. Regresyon için tahminlerin ortalaması alınır.
\end{itemize}

\subsection{Hiperparametreler}
\begin{table}[h]
\centering
{\scriptsize\renewcommand{\arraystretch}{0.4}
{\resizebox*{\linewidth}{0.4\textwidth}{
\begin{tabular}{|p{3cm}|p{1cm}|p{1cm}|p{6cm}|}
\hline
Parametre & Type & Default & Açıklama \\ \hline
n\_estimators & int & 100 & Oluşturulacak ağaç sayısı. \\ \hline
max\_depth & int & None & Oluşturulacak ağaçların maksimum derinliği. Fazla olması modelin karmaşıklığını ve aşırı uyumu artırabilir. \\ \hline
min\_samples\_split & int & 2 & Bir iç düğümün ikiye bölünmeden önce kaç örneğe sahip olması gerektiği. \\ \hline
min\_samples\_leaf & int & 1 & Bir yaprak düğümünün en az kaç örneğe sahip olması gerektiği. \\ \hline
max\_features & "sqrt", "log2", float & "sqrt" & Her bir ağaçta kullanılacak maksimum özellik sayısı.  sqrt, len(n\_features). log2, log2(n\_features) \\ \hline
bootstrap & bool & None & Bootstrap örneklemlerinin kullanılıp kullanılmayacağını belirler. \\ \hline

\end{tabular}
}}}
\end{table}

\newpage