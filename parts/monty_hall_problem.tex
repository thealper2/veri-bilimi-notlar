\section{Monty Hall Problemi}

İsmini Amerikalı televizyon sunucusu Monty Hall'den alır. Problem şu şekildedir;

\begin{enumerate}
	\item Bir yarışma programında yarışmacının önünde üç kapı vardır. Bu kapılardan birinin ardında araba, diğer ikisinin ardında ise keçiler vardır.
	\item Yarışmacı bir kapı seçer.
	\item Sunucu (Monty Hall), yarışmacının seçmediği kapılardan arkasında keçi olan birini açar.
	\item Sunucu, yarışmacıya seçimi değiştirme fırsatı sunar.
\end{enumerate}

\subsection{Problem Analizi}
Başlangıçta yarışmacının bir kapıyı seçmesiyle, üç kapı arasında bir seçim yapmış olur;

\begin{itemize}
	\item Yarışmacının başlangıçta araba olan kapıyı seçme olasılığı: $\frac{1}{3}$
	\item Yarışmacının başlangıçta keçi olan kapıyı seçme olasılığı: $\frac{2}{3}$
\end{itemize}

Sunucu, her zaman keçinin arkasında olduğu bilinen bir kapıyı açacağı için, yarışmacının ilk seçimi;

\begin{enumerate}
	\item \textbf{Yarışmacı arabanın olduğu kapıyı seçtiyse}, diğer iki kapıda keçi var. Yarışmacı kapıyı değiştirse keçi alır. \textbf{Olasılık}: $\frac{1}{3}$
	\item \textbf{Yarışmacı keçinin olduğu kapıyı seçtiyse}, diğer iki kapıdan birinde keçi var. Sunucu, kalan keçili kapıyı açar. Yarışmacı kapıyı değiştirirse arabayı kazanır. \textbf{Olasılık}: $\frac{2}{3}$
\end{enumerate}

Bu nedenle yarışmacı kapısını değiştirdiğinde kazanma olasılığı $\frac{2}{3}$ iken, değiştirmediğinde kazanma olasılığı $\frac{1}{3}$'tür.

\newpage