\section{Locality Sensitive Hashing (LSH)}

LSH, büyük veri kümeleri üzerinde yakın komşu (nearest neighbor) aramayı hızlandırmak amacıyla geliştirilmiştir. Yüksek boyutlu veri kümelerinde benzer öğeleri hızlıca bulmak, LSH algoritmasının temel amacıdır. Bu algoritma, geleneksel hashing yöntemlerinden farklı olarak, benzer öğelerin aynı hash değerine sahip olma olasılığını artırır. Böylece benzer öğeler, birbirine yakın konumlara yerleştirilir ve bu sayede yakın komşular hızlı bir şekilde bulunabilir.

LSH, verilerin belirli bir benzerlik metriği kullanılarak hash'lenmesi prensibine dayanır. Temel fikir, hash fonksiyonlarının kullanılması ve bu fonksiyonların benzer veri noktalarını aynı hash değerine veya aynı gruba yerleştirme olasılığının yüksek olmasıdır. Tek bir hash fonksiyonu kullanıldığında hatalar veya veri kayıpları yaşanabilir. Bu yüzden LSH algoritmasında genellikle birden fazla hash fonksiyonu kullanılır. Her bir veri noktası, birden fazla hash fonksiyonu kullanılarak farklı hash tablolarında saklanır



\newpage