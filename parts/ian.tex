\section{Interactive Attention Network (IAN)}
Duygu analizi, metin sınıflandırma problemlerinde kullanılır. IAN, iki ana dikkat mekanizmasını içerir:
\begin{itemize}
	\item \textbf{Cümle Dikkati (Sentence-Level Attention)}: Her bir kelimenin önemini belirlemek için cümle düzeyinde dikkat mekanizması kullanılır.
	\item \textbf{Etkileşimli Dikkat (Interactive Attention)}: Duygu analizinde, hedef kelimeler ve bağlam arasındaki etkileşimi modellemek için kullanılır.
\end{itemize}

\subsection{Çalışma Adımları}
\begin{enumerate}
	\item Metin verileri, kelime gömme katmanları ile vektör temsillerine dönüştürülür.
	\item Bu vektörler, LSTM veya RNN katmanları ile işlenir ve her bir kelimenin ardışık bağlamı öğrenilir.
	\item Cümle düzeyinde dikkat mekanizması, her bir kelimenin önem derecesini belirler.
	\item Hedef kelimeler ve bağlam arasındaki etkileşimler, etkileşimli dikkat mekanizması kullanılarak modellenir.
	\item Dikkat katmanlarından elde edilen özellikler birleştirilir.
	\item Sınıflandırıcı katman ile nihai sonuç üretilir.
\end{enumerate}

\newpage