\section{Negative Sampling}
Word embedding modelleri gibi yoğun doğrusal olmayan modellerin eğitimi sırasında sıklıkla kullanılan bir öğrenme stratejisidir. Bu strateji, eğitim veri setindeki her örneği tamamen incelemek yerine, yalnızca küçük bir alt kümesini seçerek hesaplama maliyetini azaltır. Negative sampling, özellikle yüksek boyutlu veri setleri üzerinde eğitilen word embedding (kelime gömme) modelleri için önemlidir. Word2vec gibi modellerde kullanılır. Bu tür modeller, milyonlarca kelime vektörünü öğrenmek için büyük veri setlerine ihtiyaç duyarlar ve bu, hesaplama maliyetini artırabilir. Temel fikir, her eğitim örneğinin pozitif örneği ile birlikte rastgele seçilen bir dizi negatif örneğin birleştirilmesidir. Model, doğru kelimeyi doğru tahmin etmek için olumlu örnekler üzerinde eğitilirken, yanlış tahminleri yapabilmek için negatif örnekler üzerinde de eğitilir. Bu, modelin daha genel bir bağlamda daha iyi öğrenmesini sağlar.

\subsection{Python Kodu}

\begin{lstlisting}[language=Python]
from gensim.models import Word2Vec
from gensim.models.word2vec import LineSentence
from gensim.models.callbacks import CallbackAny2Vec

# Egitim veri setinin yolu
data_path = "data.txt"

# Egitim veri setini yukleme
sentences = LineSentence(data_path)

model = Word2Vec(
    sentences,
    sg=1,  # Skip-gram modeli kullan
    window=5,  # Ilgili kelimenin baglamini belirlemek icin kullanilacak kelime penceresinin boyutu
    negative=5,  # Negative sampling icin kullanilacak negatif ornek sayisi
    workers=4,  # Paralel islemler icin kullanilacak isci sayisi
)

# Modeli kaydetme
model.save("word2vec_model")
\end{lstlisting}

\newpage
