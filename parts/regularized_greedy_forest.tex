\section{Regularized Greedy Forest}
Normal RF'den hızlıdır. Daha fazla parametreye sahiptir. İki temel bileşenden oluşur:

\begin{itemize}
    \item \textbf{Regularization (Düzenleme):} Her bir karar ağacının büyüklüğünü ve karmaşıklığını kontrol etmek için bir düzenleme yöntemi kullanır. Bu düzenleme, her bir ağacın büyüklüğünü sınırlayarak overfittingi önler.
    \item \textbf{Greedy Learning:} Her bir ağaç, bir özellik seçerek ve bu özelliğe göre bir bölme yaparak oluşturulur. Her bir bölme için en iyi özelliği seçmek için bir düzgünleştirme işlemi uygular. Bu da daha iyi genelleme sağlar.
\end{itemize}

\subsection{Python Kodu}

\begin{lstlisting}[language=Python]
from rgf import RGFClassifier

model = RGFClassifier(max_leaf=150, algorithm='RGF_Sib', test_interval=100)
model.fit(X_train, y_train)
\end{lstlisting}

\subsection{Hiperparametreler}
\begin{table}[h]
\centering
{\scriptsize\renewcommand{\arraystretch}{0.4}
{\resizebox*{\linewidth}{0.2\textwidth}{
\begin{tabular}{|p{2cm}|p{2cm}|p{1cm}|p{5cm}|}
\hline
Parametre & Type & Default & Açıklama \\ \hline
algorithm & "RGF", "RGF\_Opt", "RGF\_Sib" & "RGF" & "RGF", "l2" düzenlileştirme, "RGF\_Opt" min-penalty, "RGF\_Sib" min-penalty + sum-to-zero sibling. \\ \hline
loss & "LS", "Expo", "Log" & "LS" & LS square loss, Expo exponential loss, Log logistic loss. \\ \hline

\end{tabular}
}}}
\end{table}

\newpage