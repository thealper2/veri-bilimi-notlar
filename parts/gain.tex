\section{GAIN (Generative Adversarial Imputation Networks)}
Veir kümesindeki eksik değerleri doldurmak için geliştirilmiş GAN tabanlı bir modeldir. GAN mimarisini kullanarak eksik verileri gerçek verilere benzer şekilde tahmin etmeyi hedefler. GAN'da olduğu gibi iki ana bileşeni vardır;

\begin{itemize}
	\item \textbf{Generator (G)}: Eksik verileri doldurmak için tahminlerde bulunur.
	\item \textbf{Discriminator (D)}: Gerçek verilerle üretici tarafından üretilen tahminleri ayırt etmeye çalışır.
\end{itemize}

\subsection{Çalışma Adımları}
\begin{enumerate}
	\item Eksik veri seti X ve mask matrisi M ile başlanır.
	\item Üretici, mask matrisi ve girdi veri matrisi X kullanarak eksik yerleri tahmin eder.
	\item Ayrıştırıcı, üretici tarafından doldurulmuş veri ve gerçek veri arasında ayrım yapmaya çalışır.
	\item Hem üretici hem ayrıştırıcı için kayıp hesaplanır.
	\item Ağırlıklar güncellenir.
	\item Adımlar tekrarlanır. 
\end{enumerate}

\newpage