\section{Bayesian Ridge}
Bayes teoremini ve bayes istatistik prensiplerine dayanarak lineer regresyon temelli bir modeldir. Bayes teoremi kullanarak olasılık dağılmını güncelleyerek parametreleri belirler.

\subsection{Çalışma Adımları}
\begin{enumerate}
    \item Bir önceki dağılım (prior distribution) belirlenir. Bu, parametrelerin olası değerleri hakkında bir öngörü sağlar.
    \item Problemi bir olasılık dağılımı olarak ele alır.
    \item Veri dağılımına bağlı olarak bir olasılık fonksiyonu (likelihood function) belirlenir. Bu fonksiyon, modelin parametrelerinin belirli bir değere sahip olma olasılığını tanımlar.
    \item Model, veriye dayalı olarak bayes teoremini kullanarak önceki dağılımları günceller.
    \item Güncellenmiş dağılımlar modelin parametreleri için bir sonraki dağılımı (posterior distribution) oluşturur.
    \item Model bu dağılımı kullanarak tahmin yapar.
\end{enumerate}

\begin{table}[ht]
\centering
{\scriptsize\renewcommand{\arraystretch}{0.4}
{\resizebox*{\linewidth}{0.4\textwidth}{
\begin{tabular}{|p{2cm}|p{1cm}|p{1cm}|p{6cm}|}
\hline
Parametre & Type & Default & Açıklama \\ \hline
alpha\_1 & float & 1e-6 & Prior distribution için L2 regularizasyon hassasiyeti.  \\ \hline
alpha\_2 & float & 1e-6 & Likelihood function için L2 regularizasyon hassasiyeti.  \\ \hline
lambda\_1 & float & 1e-6 & Prior distribution için L1 regularizasyon hassasiyeti.  \\ \hline
lambda\_2 & float & 1e-6 & Likelihood function için L1 regularizasyon hassasiyeti. \\ \hline
tol & float & 1e-3 & Optimizasyon algoritmasının toleransı. \\ \hline
\end{tabular}
}}}
\end{table}

\newpage