\section{Bias Variance Tradeoff}
Bir modelin genel hatasını etkileyen iki ana bileşen olan bias ve varyans arasındaki dengenin incelenmesini ifade eder. Bu kavram, bir modelin karmaşıklığını artırırken genel hata üzerindeki etkilerini anlamaya yardımcı olur. Bias, bir modelin tahminlerinin gerçek değerlerden ne kadar uzak olduğunu ölçer. Düşük bias, modelin verilere daha iyi uyum sağladığı ve eğitim verilerine daha iyi adapte olduğu anlamına gelir. Ancak, düşük bias, modelin eğitim verilerine aşırı uyum sağlayabileceği ve yeni verilere kötü uyarlanabileceği anlamına da gelebilir (overfitting). Varyans, bir modelin farklı eğitim veri kümeleri üzerindeki tahminlerinin ne kadar değişken olduğunu ölçer. Yüksek varyans, modelin eğitim verilerine aşırı uyum sağladığı ve yeni verilere kötü uyarlandığı anlamına gelir (overfitting). Düşük varyans, modelin farklı eğitim veri kümeleri üzerinde daha tutarlı tahminler yaptığı anlamına gelir. Örneğin, karmaşık bir derin öğrenme modeli, yüksek bir varyansa sahip olabilir; çünkü eğitim verilerine çok fazla uyum sağlayabilir.

\begin{itemize}
	\item \textbf{Underfit:} yüksek bias - düşük varyans.
	\item \textbf{Overfit:} düşük bias - yüksek varyans.
	\item \textbf{Optimal:} düşük bias - düşük varyans.
\end{itemize}

\newpage