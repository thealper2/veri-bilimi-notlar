\section{Fourier Transform-Enhanced Network (FNet)}

Google tarafından 2021 yılında tanıtılan bu model, Transformer'daki hesaplama maliyetini düşürmek amacıyla, Fourier Transform kullanan yeni bir dikkat (attention) mekanizması önerir. FNet, Fourier Transform-Enhanced Network kısaltması olup, Transformer mimarisindeki pahalı dikkat (self-attention) katmanlarını Fourier Dönüşümü ile değiştirerek dikkat mekanizmasının hesaplama maliyetini düşürmeyi hedefleyen bir modeldir. Temel amacı, aynı veya daha iyi performansı, çok daha düşük hesaplama maliyeti ile sağlamaktır.

FNet, Transformer'lardaki dikkat (self-attention) mekanizmasını ortadan kaldırarak, bunu Fourier dönüşümü ile değiştirir. Bu işlem, girdinin özellik uzayında daha hızlı ve daha verimli bir dönüşüm sağlar. Fourier dönüşümü, sinyallerin zaman veya uzay alanından frekans alanına dönüştürülmesini sağlar. FNet, bu matematiksel araçtan yararlanarak modelin içindeki bilgi alışverişini dikkat mekanizmasına göre daha hızlı hale getirir. Dönüştürülen kelime temsilleri üzerinde Fourier dönüşümü uygulanır. Fourier dönüşümü, girdinin zaman veya pozisyon bilgilerini frekans uzayına taşır ve bu dönüşüm sırasında girdi bileşenleri arasındaki ilişkiler hızlı bir şekilde modellenir

\newpage