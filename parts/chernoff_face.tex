\section{Chernoff Faces}

Chernoff Faces, çok boyutlu verilerin özelliklerini farklı yüz özelliklerine bağlayan bir görselleştirme tekniğidir. İlk olarak matematikçi Herman Chernoff tarafından 1973'te tanıtılmıştır. Bu yöntem, veri setindeki her bir gözlemi bir insan yüzü olarak temsil eder ve her yüzün farklı özellikleri (göz büyüklüğü, burun uzunluğu, ağız şekli vb.) belirli değişkenlerin değerlerine bağlı olarak değişir.
Her bir gözlem (örnek), bir yüz olarak temsil edilir ve veri setindeki her bir değişken, yüzün belirli bir özelliği ile ilişkilendirilir. Örneğin:

\begin{itemize}
    \item \textbf{Gözlerin büyüklüğü}: Bir değişkenin değeri.
    \item \textbf{Ağız genişliği}: Başka bir değişkenin değeri.
    \item \textbf{Yüzün genişliği}: Diğer bir değişkenin değeri.
\end{itemize}

Yüksek boyutlu verileri hızlı bir şekilde karşılaştırma imkanı sunar. İnsan beyni yüzleri tanımada ve farklılıkları fark etmede oldukça başarılıdır. Hangi değişkenin hangi yüz özelliğiyle temsil edileceği seçimi özneldir ve veri bilimcinin kararına bağlıdır.

\newpage