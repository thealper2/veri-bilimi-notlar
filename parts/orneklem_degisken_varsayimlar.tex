\section{Örneklem, Değişken ve Varsayımları}
\subsection{İstatistik (Statistics)}
İstatistik, veri toplama, analiz etme, yorumlama ve sunma bilimidir. Gözlem ve deneylerden elde edilen verilerin incelenmesiyle, bu verilerden anlamlı sonuçlar çıkarmaya odaklanır. İki ana dalda incelenir:

\begin{itemize}
	\item \textbf{Betimsel İstatistik (Descriptive Statistics)}: Verilerin özetlenmesi ve sunulmasıyla ilgilenir. Verilerin temel özelliklerini özetler ve görsellerle açıklar. Örneğin ortalama, medyan, varyans, grafikler vb.
	\item \textbf{Çıkarımsal İstatistik (Inferential Statistics)}: Örneklem verilerine dayanarak popülasyon hakkında çıkarımlar yapılır. Küçük bir örneklem verilerinden genel popülasyon hakkında tahminler ve hipotez testleri yapılır. Örneğin hipotez testleri, güven aralıkları vb.
\end{itemize}

\subsection{Örneklem (Sample)}
Örneklem, bir popülasyonun özelliklerini temsil etmek amacıyla seçilen küçük bir veri grubudur. Popülasyon hakkında bilgi edinmek için kullanılır. Popülasyonun özelliklerini doğru bir şekilde yansıtmalıdır. Örneklemin rastgele seçilmesi her bir bireyin seçilme olasılığının eşit olmasını sağlar ve yanlılığı azaltır. Daha büyük örneklemler, popülasyonu daha iyi temsil eder.

\subsection{Değişkenler (Variables)}
Değişkenler, ölçülen veya gözlemlenen özelliklerdir. Değişkenler, her bir gözleme göre farklılık gösterir. 

\begin{itemize}
	\item \textbf{Bağımsız Değişkenler (Independent Variables)}: Diğer değişkenleri etkileyen veya tahmin eden değişkenlerdir. Araştırmacı tarafından kontrol edilir, değiştirilebilir. Bağımlı değişkende meydana gelen değişikliklerin nedeni olarak kabul edilir. Tahmin edicidir.
	\item \textbf{Bağımlı Değişkenler (Dependent Variables)}:  Bağımsız değişkenlerin etkisi altında olan değişkenlerdir. Bağımsız değişkenlerdeki değişikliklere nasıl tepki verdiği ölçülür. Araştırmacı tarafından ölçülür veya gözlemlenir. Tahmin edilendir.
\end{itemize}

\subsection{Varsayımlar (Assumptions)}
Varsayım, istatistiksel testler yapılmadan önce, verilerin ve kullanılan yöntemlerin belirli şartlara uygun olması gerektiğini belirtir.

\subsection{İstatistik Testi (Statistical Test)}
İstatistik testleri, bir hipotezin doğruluğunu değerlendirmek için kullanılan yöntemlerdir. Bu testler, verilerin belirli bir dağılıma uygun olup olmadığını veya iki veya daha fazla grubun ortalamaları arasında anlamlı bir fark olup olmadığını belirler. 

\begin{itemize}
	\item \textbf{Parametrik Testler (Parametric Tests)}: Verilerin belirli bir dağılım gösterdiğini varsayar. Örneğin t-testi, ANOVA, pearson korelasyon testi.
	\item \textbf{Nonparametrik Testler (Non-parametric Tests)}: Verilerin belirli bir dağılım göstermesi gerekmeyen testlerdir. Örneğin mann-whitney u testi, kruskal-wallis testi, spearman korelasyon testi.
\end{itemize}


\newpage