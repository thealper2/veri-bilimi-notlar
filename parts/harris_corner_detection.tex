\section{Harris Corner Detection}

Köşeleri tespit etmek için kullanılır. 1988 yılında Chris Harris ve Mike Stephens tarafından geliştirilmiştir. Harris köşe algılama algoritması, bir görüntünün küçük bölgelerinde yoğunluk değişikliklerini analiz ederek köşe noktalarını belirler. Temel prensip, bir piksellik kaydırmaların görüntüdeki nasıl yoğunluk değişikliklerine yol açtığını analiz etmektir.

\subsection{Çalışma Adımları}
\begin{enumerate}
	\item Görüntünün x ve y yönündeki gradyanları hesaplanır.
	\item Her piksel için gradyanların karesel ortalamaları ile kovaryans matrisi oluşturulur. 
	\item Harris yanıtı $R = det(M) - k * (trace(M)) ^ 2$. Burada $M$ otokovaryans matrisi, $det(M)$ matrisin determinantı, $trace(M)$ matrisin izidir.
	\item Yanıt $R$ belirli bir eşik değerinin üzerinde olan noktalar köşe olarak işaretlenir. Yerel maksimumlar tespit edilir ve bu noktalar köşe noktası olarak kabul edilir.
\end{enumerate}

\subsection{OpenCV - Harris Corner Detection}
\begin{lstlisting}[language=Python]
import cv2
import numpy as np
import matplotlib.pyplot as plt

image = cv2.imread('image.jpeg')
gray = cv2.cvtColor(image, cv2.COLOR_BGR2GRAY)
gray = np.float32(gray)
dst = cv2.cornerHarris(gray, 2, 3, 0.04)
dst = cv2.dilate(dst, None)
image[dst > 0.01 * dst.max()] = [0, 0, 255]
plt.imshow(cv2.cvtColor(image, cv2.COLOR_BGR2RGB))
plt.title('Harris Corner Detection')
plt.axis('off')
plt.show()
\end{lstlisting}

\newpage