\section{Logistic Regression}
İsmini matematikteki, lojistik (sigmoid) fonksiyondan alır. Bir bağımlı değişkenin (genellikle kategorik) olasılığını tahmin etmek veya sınıflandırmak için kullanılan bir istatistiksel modeldir. Temel amacı, bir girdi örneğini belirli bir sınıfa ait olma olasılığını tahmin etmektir. Lineer regresyon ile arasındaki fark, Lineer regreson optimum çizgiyi çizmek için "En Küçük Kareler Yöntemi (Least Squares)", lojistik regreson "Maksimum Olabilirlik (Maksimum Likelihood)" kullanır. Sigmoid Fonksiyonu, verileri 0 ile 1 arasında sıkıştırmak için kullanılan fonksiyondur.

\subsection{Avantajları}
\begin{itemize}
    \item Basit ve hızlıdır.
    \item Kolay yorumlanabilir.
    \item Olasılık tahminleri sunar.
\end{itemize}

\subsection{Dezavantajları}
\begin{itemize}
    \item Sadece doğrusal ilişkileri modelleyebilir.
    \item Overfitting riski vardır.
    \item Sadece sınıflandırma problemlerinde kullanılır.
\end{itemize}

\newpage