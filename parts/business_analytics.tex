\section{Analitik Seviyeleri}

\subsection{Descriptive (Betimleyici) Analitik}

Descriptive analitik, geçmiş verilerin analiziyle ne olduğunu anlamayı hedefler. Bu analiz yöntemi, mevcut veriyi organize eder, özetler ve genellikle raporlamalar ve veri görselleştirmeleri aracılığıyla bilgiyi sunar. Betimleyici analitik soruları cevaplar: "Ne oldu?" veya "Şu ana kadar neler yaşandı?" Descriptive analitik, işletmelere mevcut durumu anlamaları için geçmişteki performanslarını analiz etme imkanı verir. Verinin ne anlama geldiğini ve geçmişte neler yaşandığını kavramak, gelecekteki stratejileri belirlemek için önemli bir adım oluşturur.

\begin{itemize}
    \item \textbf{Raporlama}: Satış raporları, finansal tablolar, müşteri davranış raporları.
    \item \textbf{Veri Görselleştirme}: Dashboard'lar, zaman serisi grafikleri, pasta grafikleri, dağılım grafikleri.
\end{itemize}

\subsection{Predictive (Öngörücü) Analitik}

Predictive analitik, geçmiş verilere dayanarak gelecekte ne olacağını öngörmeye çalışır. Bu analiz yöntemi, makine öğrenmesi, regresyon analizleri ve veri madenciliği gibi teknikleri kullanarak olasılıksal sonuçlar sunar. Predictive analitik, "Gelecekte ne olabilir?" sorusuna yanıt arar. Predictive analitik, gelecekte karşılaşılabilecek senaryoları tahmin etmek için veri bilimciler tarafından yaygın olarak kullanılır. Bu öngörüler, karar alıcılara proaktif olma imkanı sağlar.

\begin{itemize}
    \item \textbf{Veri Madenciliği}: Metin madenciliği, web madenciliği.
    \item \textbf{Makine Öğrenmesi}: Öngörü modelleri (regresyon, zaman serisi tahmini, sınıflandırma).
    \item \textbf{Risk Analizi}: Finansal risk tahmini, müşteri segmentasyonu.
\end{itemize}

\subsection{Prescriptive (Önerici) Analitik}

Prescriptive analitik, descriptive ve predictive analitikten elde edilen verileri kullanarak, ne yapılması gerektiği konusunda önerilerde bulunur. Bu analiz yöntemi, optimizasyon teknikleri ve karar analitiği gibi ileri seviye araçlarla stratejik kararlar almaya yardımcı olur. Sorusu: "Bu durum karşısında ne yapmalıyız?". Prescriptive analitik, karar vericilere sadece olası senaryoları göstermekle kalmaz, aynı zamanda bu senaryolar karşısında en iyi eylem planını önerir.

\begin{itemize}
    \item \textbf{Karar Destek Sistemleri}: Optimizasyon modelleri, tavsiye algoritmaları, en iyi stratejik aksiyonları önerme.
    \item \textbf{Simülasyon}: Olası sonuçları simüle ederek, hangi kararın daha avantajlı olacağını gösterme.
\end{itemize}

\newpage