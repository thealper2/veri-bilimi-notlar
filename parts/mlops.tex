\section{MLOps}
MLOps, Makine Öğrenimi Operasyonları'nın (Machine Learning Operations) kısaltmasıdır. MLOps, makine öğrenimi projelerinde geliştirme, eğitim, test, dağıtım ve izleme gibi süreçlerin otomasyonunu ve standardizasyonunu sağlayan bir dizi uygulama ve yöntemler bütünüdür. MLOps, yazılım geliştirme ve işletme uygulamalarını bir araya getirerek, veri bilimi projelerinin daha hızlı, daha güvenilir ve daha ölçeklenebilir olmasını sağlar. MLOps'a neden ihtiyacımız var;
\begin{itemize}
	\item \textbf{Modellerin Üretim Ortamına Taşınması:} MLOps, modellerin üretim ortamına sorunsuz bir şekilde entegre edilmesini kolaylaştırır.
	\item \textbf{Model Performansının İzlenmesi ve Yönetilmesi:} Modellerin performansı zamanla değişebilir. Bunun sebebi `drift' tir. Drift, modelin performansının zamanla düşmesidir. İki çeşit drift bulunur. Consept drift, veri çoğaldıkça kullanılan özellikler ve hedef değişken arasındaki ilişkinin farklılaşmasıdır. Data drift, modeli eğitirken kullanılan verinin üretim mekanizmasının değişmesidir.
	\item \textbf{Otomosyon:} Model yaşam döngüsünü otomatikleştirerek zamandan tasarruf sağlar.
	\item \textbf{Güvenlik ve Uyumluluk:} Üretim ortamındaki modellerin güvenlik ve uyumluluk gereksinimlerini karşılamaya yardımcı olur.
	\item \textbf{Sürdürülebilirlik ve Ölçeklenebilirlik:} Sürekli bir geliştirme ve dağıtım döngüsü içerisindeki modellerin verimli bir şekilde yönetilmesini sağlar.
	\item \textbf{İşbirliği:} Takım içerisindeki farklı pozisyondaki kişilerin birlikte çalışmasını kolaylaştırarak iş birliğine dayalı bir süreç sunar.
\end{itemize}

\newpage