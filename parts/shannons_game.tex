\section{Shannon's Game}
Bilgi teorisinin kurucusu Claude Shannon tarafından önerilen bir kavramdır. Bir dildeki belirsizliği ve tahmin edilebilirliği anlamak için kullanılır. Metindeki kelimelerin sırasını tahmin etmeye dayanır. Shannon's Game, bir kişinin metindeki bir sonraki kelimeyi tahmin etmeye çalıştığı bir oyundur. Oyun, dilin ne kadar tahmin edilebilir olduğunu ölçmeyi amaçlar. Bu oyun sayesinde, bir dildeki bilgi miktarı, dildeki bilginin sıkıştırılabilirliğini ve entrosini analiz etmiştir.

\begin{itemize}
	\item \textbf{zero-order approximation}: Harflerin sırası birbirinden bağımsızdır.
	\item \textbf{first-order approximation}: Harflerin dildeki dağılımına göre meydana gelir.
	\item \textbf{second-order approximation}: Bir harfin görülme olasılığı kendinden önceki bir harfe bağlıdır.
	\item \textbf{third-order approximation}: Bir harfin görülme olasılığı kendinden önceki iki harfe bağlıdır.
\end{itemize}

\newpage