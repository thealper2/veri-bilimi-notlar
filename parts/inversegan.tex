\section{InverseGAN}

InverseGAN, bir modelin iç yapısını veya öğrenilmiş temsillerini anlamak ve tersine mühendislik yapmak için kullanılır. Geleneksel GAN yapısının aksine, InverseGAN'da hedef, bir modelin çıktılarından yola çıkarak giriş bilgilerine geri dönmektir. Bu yaklaşım, modelin güvenlik açıklarını belirlemek ve sistemin savunma mekanizmalarını güçlendirmek için kullanılır. InverseGAN'da üretici, çıktıların arkasındaki gizli değişkenleri (latent variables) keşfetmeye çalışır. Discriminator ise, üretilen geri dönüşlerin ne kadar gerçekçi olduğunu ölçer. Burada amaç, girdi bilgilerinin ne kadr iyi tahmin edilebileceğidir.

\newpage