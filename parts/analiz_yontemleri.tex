\section{Analiz Yöntemleri}
\subsection{Cohort Analizi}
Cohort analizi, bir işletmenin müşteri davranışlarını ve performansını daha iyi anlamak için kullanılan bir veri analizi yöntemidir. Bu yöntem, benzer özelliklere sahip müşteri gruplarını (cohort) oluşturarak, bu grupların zaman içindeki davranışlarını izlemeyi sağlar. Cohort analizinin temel amacı, belirli bir dönemde birlikte başlayan ve benzer özelliklere sahip müşteri gruplarını tanımlamak ve bu grupların performansını izlemek için kullanılır. Cohort analizi, özellikle müşteri sadakati, terk oranları ve gelir analizi gibi konularda işletmelere değerli bilgiler sunar.

\subsubsection{Python Kodu}

\begin{lstlisting}[language=Python, caption=Cohort analizi örneği]
import pandas as pd

# Ornek bir veri cercevesi olusturalim
data = {'Tarih': ['2023-01-01', '2023-01-01', '2023-02-01', '2023-02-01'],
        'MusteriID': [1, 2, 3, 4],
        'Gelir': [100, 150, 200, 50]}

df = pd.DataFrame(data)

# Musteri kaydi tarihine gore cohort olusturma
df['Cohort'] = df.groupby('Tarih')['MusteriID'].transform('min')

# Cohort analizi icin gruplama
cohorts = df.groupby(['Cohort', 'Tarih'])['Gelir'].sum().unstack()

# Her cohort'in ilk aydaki performansini hesaplama
cohort_size = cohorts.iloc[:, 0]
cohorts = cohorts.divide(cohort_size, axis=0)

# Sonuclari yazdirma
print(cohorts)
\end{lstlisting}

\newpage

\subsection{RFM Analizi}
RFM analizi, bir işletmenin müşterilerini segmentlere ayırmak ve müşteri davranışını daha iyi anlamak amacıyla kullanılan bir pazarlama analizi yöntemidir.

\begin{itemize}
    \item \textbf{Recency (Yenilik):} Müşterinin son aktivitesinin (örneğin, son satın alma) ne kadar yakın zamanda gerçekleştiğini belirtir. Genellikle bu süre içindeki geçmiş tarih ile analiz yapılır.
    \item \textbf{Frequency (Sıklık):} Müşterinin belirli bir zaman diliminde (örneğin, son bir yıl) yaptığı toplam işlem veya etkileşim sayısını belirtir. Bu sıklık, müşterinin işletmeye ne kadar sık geri döndüğünü yansıtır.
    \item \textbf{Monetary (Parasal Değeri):} Müşterinin toplam gelir veya harcama miktarını belirtir. Bu, müşterinin işletmeye ne kadar para harcadığıını yansıtır.
\end{itemize}

\subsubsection{Python Kodu}

\begin{lstlisting}[language=Python, caption=RFM analizi örneği]
import pandas as pd

# Ornek bir veri cercevesi olusturalim
data = {'MusteriID': [1, 2, 3, 4, 5],
        'Tarih': ['2023-09-10', '2023-10-15', '2023-08-20', '2023-07-05', '2023-09-25'],
        'Miktar': [100, 150, 200, 50, 300]}

df = pd.DataFrame(data)
df['Tarih'] = pd.to_datetime(df['Tarih'])

# Recency hesaplama
max_date = df['Tarih'].max()
df['Recency'] = (max_date - df['Tarih']).dt.days

# Frequency hesaplama
frequency = df.groupby('MusteriID')['Tarih'].count().reset_index()
frequency.rename(columns={'Tarih': 'Frequency'}, inplace=True)
df = df.merge(frequency, on='MusteriID', how='left')

# Monetary hesaplama
monetary = df.groupby('MusteriID')['Miktar'].sum().reset_index()
monetary.rename(columns={'Miktar': 'Monetary'}, inplace=True)
df = df.merge(monetary, on='MusteriID', how='left')

# RFM puanlarini hesaplama
# Isletmenizin kendi puanlama ve segmentasyon kriterlerini tanimlamalisiniz.

# Ornek: Recency icin, daha dusuk degerler daha iyi oldugunu varsayalim.
df['RecencyScore'] = pd.qcut(df['Recency'], q=5, labels=False)

# Ornek: Frequency ve Monetary icin, daha yuksek degerler daha iyi oldugunu varsayalim.
df['FrequencyScore'] = pd.qcut(df['Frequency'], q=5, labels=False)
df['MonetaryScore'] = pd.qcut(df['Monetary'], q=5, labels=False)

# RFM toplam puanini hesaplama
df['RFMScore'] = df['RecencyScore'] + df['FrequencyScore'] + df['MonetaryScore']

# Sonuclari goruntuleme
print(df)
\end{lstlisting}

\newpage

\subsection{CRM Analizi}
CRM Analizi (Customer Relationship Management Analysis), müşteri ilişkileri yönetimi süreçlerini incelemek, geliştirmek ve optimize etmek için kullanılan bir veri analizi yöntemidir. CRM Analizi, müşteri ilişkilerini daha iyi anlamak, müşteri memnuniyetini artırmak, satışları artırmak ve işletme karlılığını geliştirmek amacıyla kullanılır.

\subsubsection{Python Kodu}

\begin{lstlisting}[language=Python]
import pandas as pd

# Ornek bir musteri veri cercevesi olusturalim
data = {'Musteri_ID': [1, 2, 3, 4, 5],
        'Musteri_Adi': ['Alice', 'Bob', 'Charlie', 'David', 'Eve'],
        'Satin_Alma_Miktari': [100, 150, 200, 50, 300],
        'Sikayet_Sayisi': [2, 0, 1, 3, 0],
        'Ziyaret_Sikligi': [5, 3, 8, 2, 10]}

df = pd.DataFrame(data)

# Ornek bir CRM analizi yapalim
# Isletmenize ozgu analiz kriterlerinizi ve hedeflerinizi burada belirlemeniz gerekmektedir.

# Ornek: Musteri sadakatini degerlendirmek icin "Satin_Alma_Miktari" ve "Ziyaret_Sikligi" degerlerini kullanalim
df['Sadakat_Skoru'] = (df['Satin_Alma_Miktari'] * 0.6) + (df['Ziyaret_Sikligi'] * 0.4)

# Ornek: Sikayet sayisini inceleyerek hizmet kalitesini degerlendirelim
df['Hizmet_Kalitesi'] = df['Sikayet_Sayisi'].apply(lambda x: 'Kotu' if x > 2 else 'Iyi')

# Sonuclari goruntuleme
print(df)
\end{lstlisting}

\newpage

\subsection{CLV Analizi}

CLV (Customer Lifetime Value), bir işletmenin müşterilerinin yaşam boyu gelirini tahmin etmek için kullanılan önemli bir metriktir. CLV, bir müşterinin işletmeyle ilişkisi boyunca yarattığı geliri hesaplayarak, işletmelere, müşterilerini daha iyi anlama, pazarlama stratejilerini optimize etme ve müşteri sadakatini artırma konularında yardımcı olur. CLV Formülü:

\[\text{CLV} = \frac{\text{Musterinin Ortalama Satin Alma Degeri * Satin Alma Sikliği * Musterinin Omru}}{\text{İndirim Orani}}\]

Bu formülde:
\begin{itemize}
    \item \textbf{Müşterinin Ortalama Satın Alma Değeri:} Bir müşterinin ortalama bir satın alma işlemi sırasında harcadığı miktar.
    \item \textbf{Satın Alma Sıklığı:} Bir müşterinin belirli bir zaman diliminde (örneğin, bir yıl) ne sıklıkta satın alma işlemi yaptığı.
    \item \textbf{Müşterinin Ömrü:} Bir müşterinin işletme ile ilişkisi boyunca geçirdiği süre (aylar veya yıllar cinsinden).
    \item \textbf{İndirim Oranı:} İşletme tarafından verilen indirimler veya kampanyaların etkisiyle düzeltilen bir faktör. Bazı müşteriler daha fazla indirim alabilir.
\end{itemize}

\subsubsection{Python Kodu}

\begin{lstlisting}[language=Python, caption=CLV analizi örneği.]
import pandas as pd

# Ornek bir musteri veri cercevesi olusturalim
data = {'Musteri_ID': [1, 2, 3, 4, 5],
        'Musteri_Adi': ['Alice', 'Bob', 'Charlie', 'David', 'Eve'],
        'Toplam_Harcama': [1000, 1500, 2000, 500, 3000],
        'Satin_Alma_Sikligi': [5, 3, 8, 2, 10],
        'Musteri_Omru': [24, 36, 60, 12, 48]}

df = pd.DataFrame(data)

# Ortalama satin alma degeri hesaplama
df['Ortalama_Satin_Alma_Degeri'] = df['Toplam_Harcama'] / df['Satin_Alma_Sikligi']

# CLV hesaplama
indirim_orani = 0.1  # Ornek bir indirim orani
df['CLV'] = (df['Ortalama_Satin_Alma_Degeri'] * df['Satin_Alma_Sikligi'] * df['Musteri_Omru']) / (1 + indirim_orani)

# Sonuclari goruntuleme
print(df)
\end{lstlisting}

\newpage