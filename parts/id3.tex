\section{ID3}
1986 yılında Ross Quinlan tarafından oluşturulmuştur. Veri setindeki öznitelikleri kullanarak bir karar ağacı oluşturur ve bu ağaç ile sınıflandırma yapar. ID3, kategorik değerlerle çalışabilir fakat sayısal değerleri doğrudan işleyemez.

\subsection{Çalışma Adımları}
\begin{enumerate}
    \item Öznitelikler arasından en bilgilendirici olanları bulmak için bilgi kazancı (information gain) kullanır.
    \item Seçilen en bilgilendirici öznitelik, bir karar ağacının bir düğümü olarak kullanılır. Her düğüm, bu özniteliğin farklı değerlerine göre alt düğümlere bölünür.
    \item ID3, özniteliklerin dallanma noktalarını ve karar ağacındaki düğümleri oluşturur. Her bir düğüm, bir öznitelik ve bu özniteliğin değerlerine göre alt düğümlere bölünür.
    \item ID3, bu işlemi her alt düğüm için tekrarlar. Her bir alt düğüm, bir veri kümesinin daha homojen alt kümelerine bölünmesini sağlar.
    \item ID3, bu işlemi veri seti tamamen sınıflandırıldığında durdurur ve bir karar ağacı elde eder.
\end{enumerate}

\subsection{Python Kodu}

\begin{lstlisting}[language=Python]
from id3 import Id3Estimator

classifier = Id3Estimator()
classifier.fit(X_train, y_train)
\end{lstlisting}

\newpage
