\section{State-of-the-art DL Mimarileri}
State-of-art modeller, derin öğrenmede çeşitli görevler üzerinde en iyi performansı elde eden en gelişmiş sinir ağı mimarilerini ifade eder. Avantajları:
\begin{itemize}
	\item \textbf{Yüksek Performans}
	\item \textbf{Ölçeklenebilirlik}
	\item \textbf{Verimlilik}
	\item \textbf{Yenilikçilik}
	\item \textbf{Esneklik}
\end{itemize}

\subsection{InceptionV3}
2016 yılında Szgedy ve diğerleri tarafından tanıtılmıştır. InceptionV1'in geliştirilmiş versiyonudur. 47 katmandan oluşur. ImageNet veri setinde hata oranı \%3.46'dır. Yaklaşık 24 milyon eğitilebilir parametreye sahiptir.

\subsection{VGG16 (Visual Geometry Group - 16)}
Simonyan ve Zisserman tarafından tanıtılmıştır. 2014 yılında ImageNet yarışmasında hata oranı \%7.3'dir. 16 katmandan oluşur. Yaklaşım 130 milyon parametreye sahiptir. 13 adet 3x3 Conv, 2x2 MaxPool oluşur. 

\subsection{VGG19 (Visual Geometry Group - 19)}
VGG16'ya 3 tane ek Conv katmanı eklenmiş halidir. Yani toplamda 16 dönüşüm katmanı vardır. Yaklaşım 130 milyon parametreye sahiptir.

\subsection{ResNet50}
Kaybolan gradyanlar problemini çözmek için üretilmiştir. Bağlantı atlama adı verilen bir teknik kullanarak ağın birkaç eğitim katmanını atlayarak doğrudan çıkışa bağlanmasını sağlar.
48 conv, 1 MaxPool, 1 AvgPool olmak üzere toplam 50 adet katmandan oluşur. 26 milyon parametreye sahiptir. ImageNet yarışmasında hata oranı \%5.1'dir. 

\subsection{Xception}
2017 yılında Chollet tarafından tanıtılmıştır. Inception modelinden ilham alır. 71 Conv katmanı vardır. ImageNet yarışmasında hata oranı \%5.6'dır.

\newpage