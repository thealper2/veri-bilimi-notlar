\section{Simpson's Paradox}

İki veya daha fazla grubun analizinde ortaya çıkan bir olgudur. Bir veri kümesindeki grupların ayrı ayrı analiz edildiğinde gözlemlenen ilişkinin, gruplar birleştirildiğinde tersine dönebileceğini ifade eder. 
Bir hastanede iki farklı tedavi yöntemi (A ve B) üzerine yapılan bir araştırmayı ele alalım:
\begin{itemize}
    \item \textbf{Kadınlar}: Tedavi A \%90 başarı, Tedavi B \%80 başarı.
    \item \textbf{Erkekler}: Tedavi A \%70 başarı, Tedavi B \%60 başarı.
\end{itemize}

Bu durumda, hem kadınlar hem de erkekler için Tedavi A daha başarılı görünmektedir. Ancak, tüm hastalar bir arada değerlendirildiğinde, Tedavi B'nin genel başarı oranı Tedavi A'dan daha yüksek olabilir, çünkü tedavi seçimi hastaların sağlık durumları, yaş grupları veya diğer faktörler gibi değişkenlerle çelişebilir. Bu, veri segmentasyonunun doğru yapılmasının önemini vurgular.

\newpage