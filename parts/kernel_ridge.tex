\section{Kernel Ridge}
Doğrusal olmayan regresyon problemlerini çözmek için kullanılır. Temelde, Ridge fonksiyonu ile aynıdır. Sadece düzenleme (regularization) işleminden önce doğrusal olmayan ilişkileri modellemek için özellik vektörlerini doğrusal olmayan bir şekilde dönüştürür. Bu dönüşüm işlemi için çekirdek fonksiyonlarını kullanır.

\subsection{Hiperparametreler}
\begin{table}[h]
\centering
{\scriptsize\renewcommand{\arraystretch}{0.4}
{\resizebox*{\linewidth}{0.4\textwidth}{
\begin{tabular}{|p{3cm}|p{1cm}|p{1cm}|p{6cm}|}
\hline
Parametre & Type & Default & Açıklama \\ \hline
alpha & float & 1 & Ridge regresyonundaki düzenleme parametresi. \\ \hline
kernel & str & "linear" & Kullanılacak olan çekirdek fonksiyonu. \\ \hline
gamma & float & None & RBF ve Polinom çekirdeklerinde kullanılan çekirdek parametresidir. Çekirdek fonksiyonunun esnekliğini belirler. \\ \hline
degree & float & 3 & Polinom çekirdeklerinde kullanılır. Polinomun derecesini belirler. \\ \hline

\end{tabular}
}}}
\end{table}

\newpage