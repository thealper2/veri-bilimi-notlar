\section{Adaptive Boosting}
Zayıf öğrenicileri bir araya getirerek güçlü bir öğrenici oluşturmak için kullanılan bir toplu öğrenme yöntemidir.

\subsection{Çalışma Adımları}
\begin{enumerate}
    \item Her bit örnek başlangıçta 1/N ağırlığına sahiptir. N, toplam örnek sayısı.
    \item Bir zayıf öğrenici eğitilir.
    \item Zayıf öğrenicinin doğruluğu hesaplanır.
    \item Doğruluğu düşük olan örneklerin ağırlığı arttırılır. Bu, yanlış sınıflandırılan örneklerin daha fazla dikkate alınmasını sağlar. Modelin bu örnekleri daha iyi sınıflandırması beklenir.
    \item Ağırlıklandırılmış eğitim veri seti, bir sonraki zayıf öğreniciyi eğitmek için kullanılır. Bu süreç, belli bir iterasyon (boosting round) için tekrarlanır.
    \item Her bir zayıf öğrenici, tahminlerin gücüne göre bir ağırlıkla birleştirilir. Bu ağırlıklandırma işlemi, her bir öğreticinin doğruluğuna göre belirlenir. Daha doğru tahminler, daha büyük ağırlığa sahip olur.
    \item Zayıf öğrenicilerin bir araya getirilmesiyle güçlü bir öğrenici elde edilir.
\end{enumerate}

\subsection{Hiperparametreler}

\begin{table}[h]
\centering
{\scriptsize\renewcommand{\arraystretch}{0.4}
{\resizebox*{\linewidth}{0.3\textwidth}{
\begin{tabular}{|p{2cm}|p{1cm}|p{1cm}|p{6cm}|}
\hline
Parametre & Type & Default & Açıklama \\ \hline
estimator & object & None & Kullanılacak zayıf öğrenici türü. \\ \hline
n\_estimators & float & 1e-6 & Kullanılacak zayıf öğrenici sayısı. Fazla olması karmaşıklığı ve overfit'i artırabilir.  \\ \hline
learning\_rate & float & 1e-6 & Her bir zayıf öğrenicinin katkısını artırmak veya azaltmak için kullanılan parametre.  \\ \hline
loss & float & 1e-6 & Kayıp fonksiyonu.  \\ \hline
random\_state & bool & False & Rastgelelik kontrolü. \\ \hline
\end{tabular}
}}}
\end{table}

\newpage