\section{Focal Modulation}

Focal Modulation, dikkat mekanizmalarının karmaşıklığını azaltmak ve aynı anda hem yerel hem de küresel (global) bilgiyi işleyebilen bir yapı sağlamak için geliştirilmiş bir yaklaşımdır. Bu yöntem, girdideki her noktayı ya da özelliği işlemekte "odaklama" ve "modülasyon" adımlarını kullanır. Temel olarak, girdideki her öğe üzerinde hem yerel (local) hem de küresel (global) bilgiyi birleştirir ve bu şekilde daha etkili bir bilgi işleme sağlar. 

Focal Modulation, iki ana bileşen üzerine kuruludur: odaklama (focal) ve modülasyon (modulation). Bu iki bileşen, hem yerel hem de küresel bilgiyi birleştirerek çalışır.

\begin{itemize}
    \item \textbf{Odaklanma (Focal)}: Model, her bir öğeyi yerel olarak inceler ve bu öğenin çevresindeki komşu öğelerle olan ilişkisini değerlendirir. Bu adım, dikkat mekanizmalarında olduğu gibi her bir öğe için bir dikkat skorunun hesaplanmasına benzer. Ancak burada dikkat başlıkları kullanılmaz. Bunun yerine, daha basit ve doğrudan bir odaklama işlemi yapılır.
    \item \textbf{Modülasyon (Modulation)}: Modülasyon adımı, her bir öğenin küresel bağlamda nasıl işleneceğini belirler. Yerel odaklama ile elde edilen bilgiler, küresel bilgi ile birleştirilir ve her öğe için daha güçlü bir temsil elde edilir. Bu adım, girdinin küresel bağlamı ile yerel bilgiyi bir araya getirir ve son temsili oluşturur.
\end{itemize}

\newpage