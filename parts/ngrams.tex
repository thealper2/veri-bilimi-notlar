\section{N-Grams}
Ngramlar, bir metin veya serideki ardışık "n" elemanlık gruplardır. Bu gruplar, metindeki belirli bir örüntüyü veya dilbilgisel yapının bir bölümünü temsil eder. N-gram modelinin doğruluğu seçilen n değerine bağlı olarak değişebilir. N-gram olasılıkları kullanılarak bir metin tamamlama yapılabilir. 
\begin{itemize}
    \item \textbf{Uni-gram (1-gram):} Tek bir kelime veya karakteri temsil eder. Her bir kelimenin veya karakterin bağımsız olarak ele alındığı durumlarda faydalıdır.
    \item \textbf{Bi-gram (2-gram):} Ardışık iki kelime veya karakteri temsil eder. Kelimeler arasındaki bağlantıları veya ilişkileri analiz etmek için faydalıdır.
    \item \textbf{Tri-gram (3-gram):} Ardışık üç kelime veya karakteri temsil eder.
\end{itemize}

\[\text{Bir kelimenin başka bir kelimenin ardından gelme olasılığı} = \frac{\text{birlikte geçme sayısı}}{\text{İlk kelimenin geçme sayısı}}\]

\begin{lstlisting}[language=Python]
import nltk
from nltk.util import ngrams
from nltk.tokenize import word_tokenize

# Ornek metin
text = "Bu bir ornek cumle."

# Metni kelimelere ayir
tokens = word_tokenize(text)

# Bi-gramlar
bi_grams = list(ngrams(tokens, 2))
print("Bi-gramlar:", bi_grams)
# Bi-gramlar: [('Bu', 'bir'), ('bir', 'ornek'), ('ornek', 'cumle'), ('cumle', '.')]

# Tri-gramlar
tri_grams = list(ngrams(tokens, 3))
print("Tri-gramlar:", tri_grams)
# Tri-gramlar: [('Bu', 'bir', 'ornek'), ('bir', 'ornek', 'cumle'), ('ornek', 'cumle', '.')]
\end{lstlisting}

\newpage