\section{Conditional Random Fields (CRF)}
CRF, dizisel veri ve yapılandırılmış çıktı problemleri için kullanılan bir olasılıksal grafik modelidir. Etiketlerin birbirine bağımlı olduğu durumlarda etiket dizilerinin tahmin edilmesi için güçlü bir modeldir. Etiketleme problemlerinde, gözlemler verildiğinde bu gözlemler üzerinde etiketlerin olasılık dağılımını modellemeye çalışır. Genellikle POS Tagging, NER gibi NLP problemlerinde kullanılır. CRF, kenar potansiyelleri ve düğüm potansiyelleri üzerinden modelleme yapar.
\begin{itemize}
	\item Kenar potansiyelleri, ardışık etiketler arasındaki geçiş olasılıklarını temsil eder.
	\item Düğüm potansiyelleri, her bir gözlem-etiket çiftinin olasılıklarını temsil eder.
\end{itemize}

\newpage